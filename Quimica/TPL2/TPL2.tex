\documentclass[12pt,a4paper]{report}
\usepackage[left=2.5cm,right=2.5cm,top=3cm,bottom=3cm]{geometry}
\usepackage{fancyhdr}
\usepackage{etoolbox}
\usepackage{titlesec}
\usepackage{amssymb}
\usepackage{amsmath}
\usepackage{graphicx}
\usepackage{ulem}
\usepackage{cancel}
\usepackage{enumitem}
\usepackage{lmodern}
\usepackage{tikz}
\usepackage{afterpage}

\geometry{a4paper}

% Configuración de cabecera y pie de página
\pagestyle{fancy}
\fancyhf{} 
\fancyhead[L]{UTN-FRC}
\fancyhead[C]{QUÍMICA: TPL1}
\fancyhead[R]{2R3}
\renewcommand{\headrulewidth}{0.4pt}
\fancyfoot[C]{\vfill\thepage}

% Cambio en el estilo de las páginas de capítulo
\patchcmd{\chapter}{\thispagestyle{plain}}{\thispagestyle{fancy}}{}{}

% Tamaños de fuente para matemáticas
\DeclareMathSizes{15}{13}{8}{8}

% Configuración del título del documento
\title{%
  \fontsize{25}{0}\selectfont Universidad Tecnológica Nacional \\
  \fontsize{22}{30}\selectfont Química \\
  \fontsize{18}{25}\selectfont TPL1: Soluciones, Termoquimica y Cinetica Quimica.
}
\author{
  Jerónimo Billoto - 415323\\
  Luciano Cortesini - 402719\\
  Gaston Grasso - 401892\\
  Ignacio Gil - 401891\\
}
\date{23 / 08 / 2024}

% Formato de títulos y secciones
\titleformat{\chapter}[block]
  {\normalfont\huge\bfseries}{Experiencia \thechapter}{0pt}{\Huge}
\titlespacing*{\chapter}{0pt}{-30pt}{0pt}

\titleformat{\section}[block]
  {\normalfont\Large\bfseries}{\thesection}{2mm}{\Large}
\titlespacing*{\section}{0pt}{3.5ex plus 1ex minus .2ex}{2.3ex plus .2ex}

\titleformat{\subsection}[block]
  {\normalfont\large\bfseries}{}{0pt}{\large}
\titlespacing*{\subsection}{0pt}{3.25ex plus 1ex minus .2ex}{1.5ex plus .2ex}

\begin{document}
\maketitle
\section*{Cuestionario de orientacion}

\begin{enumerate}

  \item Con que tipo de material de vidrio se prepara una solucion?
    Con material volumetrico como matraces aforados, vasos de precipitados, y cilindros graduados.
  \item Cual es el objetivo de usar una propipeta?
    El objetivo es medir con precisión el volumen de un líquido y evitar el contacto directo con sustancias químicas peligrosas.
  \item Cuantos gramos de $Ca(NO_3)_2$ se necesitan para preparar 150mL de solucion 0.25M?

\begin{equation*}
\begin{aligned}
1 Mol  \, &\rightarrow 164.1g/mol \,Ca(NO_3)_2\\[6pt]
0.005 Mol  \, &\rightarrow 6.15g \,Ca(NO_3)_2
\end{aligned}
\end{equation*}

    Entonces$ \rightarrow Ca(NO_3)_2= 164.1 g/mol$

\begin{equation*}
\begin{aligned}
1 Mol  \, &\rightarrow 164.1g/mol \,Ca(NO_3)_2\\[6pt]
0.005 Mol  \, &\rightarrow 6.15g \,Ca(NO_3)_2
\end{aligned}
\end{equation*}

La respuesta es 6.15g\\
  \item que volumen de  una solucion concentrada de $HNO_3$ al $38\% $ P/P, cuya densidad es de 1,19g/mL, es necesario para preparar 500mL de una solucion de $HNO_3$ 0.8M?
 
    $$
    \begin{aligned}
    1000mL \,  &\rightarrow&& 0.8 mol \,HNO_3\\[6pt]
    500mL\,  &\rightarrow&& 0.4 \,HNO_3\, \rightarrow 25.2g HNO_3
    \end{aligned}
    $$
    $37\%$ de $1.19g/ml = 0.44g/ml$\\
    $25.2g/0.44g/ml=57.27ml$\\

\end{enumerate}

\chapter{}

\begin{enumerate}[label=\alph*]

\item Para preparar una solucion 0.010M de $KMNnO_4$ en un matraz de 50mL se necesitan: 0.079g de $KMNnO_4$.\\
  
  
  calculos:
\begin{equation*}
\begin{aligned}
1 Mol  \, &\rightarrow 158.04g \,KMnO_4\\[6pt]
0.01 Mol  \, &\rightarrow 1.5804g \,KMnO_4\\[6pt]
1000mL \,  &\rightarrow 1.5804g \,KMnO_4  \\[6pt]
50mL\,  &\rightarrow 0.079g \, KMnO_4
\end{aligned}
\end{equation*}
|
\item usamos una balanza para medir el $KMnO_4$, en vez de pesar 0.079g, pesamos 0.084g pero no hubo derrames de ningun tipo.
\item pasamos el $KMnO_4$ al vaso de precipitados y enjuagamos el vidrio de reloj con agua destilada asegurandonos de que la solucion cayera en el vaso, luego disolvimos la mayor cantidad de soluto con la varilla de vidrio.
\item Agregamos el aproximadamente 20mL de solvente en el matraz y con ayuda del embudo trasvasamos todo el material disuelto.
\item enrrasamos exactamente a 50mL
\item colocamos el tapon y agitamos el matraz suavemente hasta que casi no quedo material sin disolver. Como el soluto era de dificil solucion, esto no fue facil.

\end{enumerate}
 
\chapter{}

\begin{enumerate}[label=\alph*]
  \item Para preparar 100mL de solucion de $NaCl$ 0.05M a partir de el reactivo solido se necesitan 0.2922g de $NaCl$

  \item Se utilizó la balanza y medimos exactamente 0.2922 g de $NaCl$
  Calculos:

      $$
\begin{aligned}
   1 Mol  \, &\rightarrow&& 58.443g \,NaCl\\[6pt]
   0.05 Mol  \, &\rightarrow&& 2.922g \,NaCl\\[6pt]
    \\
   1000mL \,  &\rightarrow&& 2.922g \,NaCl\\[6pt]
    100mL\,  &\rightarrow&& 0.2922g \,NaCl\\[6pt]
\end{aligned}
$$

  \item Trasvasamos la sal al matraz utilizando el embudo y la varilla de vidrio.

  \item Enjuagamos el vidrio reloj para asegurarnos de no perder parte del material solido.

  \item Terminamos de enrasar correctamente a 100 ml.

  \item Colocamos la tapa al matraz y homogeneizamos suavemente, quedando una solucion transparente.
\end{enumerate}

\chapter{}
\begin{enumerate}[label=\alph*]
  \item Para preparar 100 mL de solucion de $HCL$ 0.075 M, se necesitan 15 ml de $HCl$ 0,5 M.

  Calculos:
      $$
\begin{aligned}
   1 Mol  \, &\rightarrow&& 36.453 \,HCl\\[6pt]
   0.075 Mol  \, &\rightarrow&&  2.733g \,HCl\\[6pt]
    \\
   1 Mol  \, &\rightarrow&& 36.453g \,HCl\\[6pt]
   0.5 Mol  \, &\rightarrow&&  18.2265g \,HCl\\[6pt]
    \\
   1000mL \,  &\rightarrow&& 2.733g \,HCl\\[6pt]
    100mL\,  &\rightarrow&& 0.2733g \,HCl\\[6pt]
    \\
   18.2265g \,HCl \,  &\rightarrow&& 1000 mL\\[6pt]
    0.2733 g \,HCl &\rightarrow&& 15 mL\\[6pt]
\end{aligned}
$$

  \item Agregamos unos mL del disolvente en el matraz, no nos fijamos correctamente que sea 1/3 del volumen total.

  \item Medimos con una pipeta el volumen de sustancia liquida necesaria para la preparacion de la solucion.

  \item Agregamos cuidadosamente el soluto en el matraz. Como no nos fijamos anteriormente de agregar 1/3 del volumen total con disolvente, sino que colocamos mas del 1/3, se nos llenó el matraz, quedando una solucion de menos de 0.075 M.

  \item Tiramos lo excedido del matraz hasta por debajo de la medida, y completamos finalmente con el disolvente hasta el enrase correcto con la pipeta.

  \item Colocamos el tapon al matraz y homogeneizamos suavemente.
\end{enumerate}

\chapter{}
\section*{Observaciones}
\begin{enumerate}[label=$\cdot$,left=1em]

  \item NaOH (hidróxido de sodio): Al agregar agua, se sintió un aumento de temperatura en el tubo de ensayo, indicando una reacción exotérmica. Esto se debe a que la disolución de NaOH en agua libera calor.

  \item NH4Cl (cloruro de amonio): Al agregar agua, no se sintió un cambio significativo de temperatura, lo que sugiere una reacción endotérmica. La disolución de NH4Cl en agua absorbe calor del entorno.

\end{enumerate}

\section*{Conclusiones}

\begin{enumerate}[label=$\cdot$,left=1em]

  \item Las reacciones químicas pueden ser clasificadas como endotérmicas o exotérmicas según si absorben o liberan calor.

  \item La disolución de NaOH en agua es una reacción exotérmica, mientras que la disolución de NH4Cl es endotérmica.

  \item Es fundamental seguir las recomendaciones de limpieza del material de vidrio y porcelana para asegurar la precisión y seguridad en el laboratorio.

\end{enumerate}

\section*{Nota Adicional}

El profesor calentó el hidróxido de sodio (NaOH) pero no el cloruro de amonio (NH4Cl), lo que pudo haber influido en la percepción de la temperatura durante la disolución.



\chapter{}

\section*{Observaciones}
\begin{enumerate}[label=$\cdot$,left=1em]

  \item Probeta con $H_2O_2$ concentrado:
  \begin{itemize}
    \item Al añadir el KI, se observó una rápida formación de burbujas y espuma, indicando una reacción rápida.
    \item La presencia de burbujas de oxígeno ($O_2$) fue evidente, lo que confirma la descomposición del $H_2O_2$.
  \end{itemize}

  \item Probeta con $H_2O_2$ diluido:
  \begin{itemize}
    \item Al añadir el KI, la formación de burbujas y espuma fue menos intensa y más lenta en comparación con la probeta con $H_2O_2$ concentrado.
    \item La reacción fue visible, pero a una velocidad menor.
  \end{itemize}

\end{enumerate}

\section*{Conclusiones}
\begin{enumerate}[label=$\cdot$,left=1em]

  \item La concentración del reactivo ($H_2O_2$) influye significativamente en la velocidad de la reacción. Una mayor concentración de $H_2O_2$ resulta en una reacción más rápida.

  \item La adición de un catalizador (KI) acelera la descomposición del $H_2O_2$, demostrando su efecto en la reducción de la energía de activación necesaria para la reacción.

  \item La reacción de descomposición del $H_2O_2$ es exotérmica, liberando oxígeno gaseoso ($O_2$) y agua ($H_2O$).

\end{enumerate}


\end{document}
