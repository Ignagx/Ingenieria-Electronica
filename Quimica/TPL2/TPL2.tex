\documentclass[12pt,a4paper]{report}
\usepackage[left=2.5cm,right=2.5cm,top=3cm,bottom=3cm]{geometry}
\usepackage{fancyhdr}
\usepackage{etoolbox}
\usepackage{titlesec}
\usepackage{amssymb}
\usepackage{amsmath}
\usepackage{graphicx}
\usepackage{ulem}
\usepackage{cancel}
\usepackage{enumitem}
\usepackage{lmodern}
\usepackage{tikz}
\usepackage{afterpage}

\geometry{a4paper}

% Configuración de cabecera y pie de página
\pagestyle{fancy}
\fancyhf{} 
\fancyhead[L]{UTN-FRC}
\fancyhead[C]{QUÍMICA: TPL1}
\fancyhead[R]{2R3}
\renewcommand{\headrulewidth}{0.4pt}
\fancyfoot[C]{\vfill\thepage}

% Cambio en el estilo de las páginas de capítulo
\patchcmd{\chapter}{\thispagestyle{plain}}{\thispagestyle{fancy}}{}{}

% Tamaños de fuente para matemáticas
\DeclareMathSizes{15}{13}{8}{8}

% Configuración del título del documento
\title{%
  \fontsize{25}{0}\selectfont Universidad Tecnológica Nacional \\
  \fontsize{22}{30}\selectfont Química \\
  \fontsize{18}{25}\selectfont TPL1: Soluciones, Termoquimica y Cinetica Quimica.
}
\author{
  Jerónimo Billoto - 415323\\
  Luciano Cortesini - 402719\\
  Gaston Grasso - 401892\\
  Ignacio Gil - 401891\\
}
\date{23 / 08 / 2024}

% Formato de títulos y secciones
\titleformat{\chapter}[block]
  {\normalfont\huge\bfseries}{Experiencia \thechapter}{0pt}{\Huge}
\titlespacing*{\chapter}{0pt}{-30pt}{0pt}

\titleformat{\section}[block]
  {\normalfont\Large\bfseries}{\thesection}{2mm}{\Large}
\titlespacing*{\section}{0pt}{3.5ex plus 1ex minus .2ex}{2.3ex plus .2ex}

\titleformat{\subsection}[block]
  {\normalfont\large\bfseries}{}{0pt}{\large}
\titlespacing*{\subsection}{0pt}{3.25ex plus 1ex minus .2ex}{1.5ex plus .2ex}

\begin{document}
\maketitle
\section*{Cuestionario de orientacion}

\begin{enumerate}

  \item Con que tipo de material de vidrio se prepara una solucion?
  \item Cual es el objetivo de usar una propipeta?
  \item Cuantos gramos de $Ca(NO_3)_2$ se necesitan para preparar 150mL de solucion 0.25M?
  \item que volumen de  una solucion concentrada de $HNO_3$ al $38\% $ P/P, cuya densidad es de 1,19g/mL, es necesario para preparar 500mL de una solucion de $HNO_3$ 0.8M?
 
\end{enumerate}

\chapter{}

\begin{enumerate}[label=\alph*]

\item Para preparar una solucion 0.010M de $KMNnO_4$ en un matraz de 50mL se necesitan: 0.079g de $KMNnO_4$.\\
  
  
  calculos:
      $$
\begin{aligned}
   1 Mol  \, &\rightarrow&& 158.04g \,KMnO_4\\[6pt]
   0.01 Mol  \, &\rightarrow&& 1.5804g \,KMnO_4\\[6pt]
    \\
   1000mL \,  &\rightarrow&& 1.5804g \,KMnO_4  \\[6pt]
    50mL\,  &\rightarrow&& 0.079g \, KMnO_4\\[6pt]
\end{aligned}
$$

\item usamos una balanza para medir el $KMnO_4$, en vez de pesar 0.079g, pesamos 0.084g pero no hubo derrames de ningun tipo.
\item pasamos el $KMnO_4$ al vaso de precipitados y enjuagamos el vidrio de reloj con agua destilada asegurandonos de que la solucion cayera en el vaso, luego disolvimos la mayor cantidad de soluto con la varilla de vidrio.
\item Agregamos el aproximadamente 20mL de solvente en el matraz y con ayuda del embudo trasvasamos todo el material disuelto.
\item enrrasamos exactamente a 50mL
\item colocamos el tapon y agitamos el matraz suavemente hasta que casi no quedo material sin disolver. como el soluto era de dificil solucion, esto no fue facil.

\end{enumerate}
 
\chapter{}

\begin{enumerate}[label=\alph*]
  \item Para preparar 100mL de solucion de $NaCl$ 0.05M a partir de el reactivo solido se necesitan 0.2922g de $NaCl$--

  \item Se
    
\end{enumerate}

\end{document}

