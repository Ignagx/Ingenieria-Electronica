\documentclass[12pt]{report}
\usepackage[left=2.5cm,right=2.5cm,top=3cm,bottom=3cm]{geometry}
\usepackage{fancyhdr}
\usepackage{etoolbox}
\usepackage{titlesec}
\usepackage{titling} % Para personalizar el título
\usepackage{graphicx}
\usepackage{hyperref}
\usepackage{amsmath}
\usepackage{indentfirst}

\geometry{a4paper}

% Configuración de cabecera y pie de página
\pagestyle{fancy}
\fancyhf{} 
\fancyhead[L]{UTN-FRC}
\fancyhead[C]{QUÍMICA: TPL1}
\fancyhead[R]{2R3}
\renewcommand{\headrulewidth}{0.4pt}
\fancyfoot[C]{\vfill\thepage}

% Cambio en el estilo de las páginas de capítulo
\patchcmd{\chapter}{\thispagestyle{plain}}{\thispagestyle{fancy}}{}{}

% Tamaños de fuente para matemáticas
\DeclareMathSizes{15}{13}{8}{8}

% Configuración del título del documento
\title{%
  \fontsize{25}{0}\selectfont Universidad Tecnológica Nacional \\
  \fontsize{22}{30}\selectfont Química \\
  \fontsize{18}{25}\selectfont TPL1: : Normas de Seguridad, Uso de Material de Vidrio, reactivo limitante y en exceso.
}
\author{
  Jerónimo Billoto - 415323\\
  Luciano Cortesini - 402719\\
  Gaston Grasso - 401892\\
  Ignacio Gil - 401891\\
}
\date{14 / 06 / 2024}

% Formato de títulos y secciones
\titleformat{\chapter}[block]
  {\normalfont\huge\bfseries}{Experiencia \thechapter}{0pt}{\Huge}
\titlespacing*{\chapter}{0pt}{-30pt}{0pt}

\titleformat{\section}[block]
  {\normalfont\Large\bfseries}{\thesection}{2mm}{\Large}
\titlespacing*{\section}{0pt}{3.5ex plus 1ex minus .2ex}{2.3ex plus .2ex}

\titleformat{\subsection}[block]
  {\normalfont\large\bfseries}{}{0pt}{\large}
\titlespacing*{\subsection}{0pt}{3.25ex plus 1ex minus .2ex}{1.5ex plus .2ex}

\begin{document}
\maketitle
\section*{Introducción}
En este informe, exploraremos tres experiencias realizadas en el laboratorio relacionadas con la química. Estudiaremos como usar las pipetas, los espectros de emisión de distintas sales metálicas y calcularemos el volumen molar de un gas. Para comprender mejor estos conceptos, revisemos brevemente cada uno de ellos:

\begin{enumerate}
  \item Pipeteo:
    \begin{itemize}
      \item El pipeteo es una técnica utilizada para medir volúmenes precisos de líquidos.
      \item Las pipetas son instrumentos de laboratorio que permiten transferir volúmenes específicos con alta precisión.
      \item El enrasado es importante para asegurar que el líquido alcance exactamente la marca de la pipeta.
    \end{itemize}
  \item Espectros de Emisión:
    \begin{itemize}
      \item Los espectros de emisión son patrones de líneas brillantes o colores específicos que se observan cuando una sustancia emite luz.
      \item Cada elemento químico tiene su propio espectro de emisión característico debido a las transiciones electrónicas en sus átomos.
      \item Estos espectros se utilizan para identificar elementos en la naturaleza.
    \end{itemize}
  \item Volumen Molar:
    \begin{itemize}
      \item El volumen molar es el volumen ocupado por un mol de sustancia a una temperatura y presión específicas.
      \item Se calcula utilizando la ecuación de los gases ideales:
      $$V = \frac{nRT}{p}$$
      Donde: 
      \begin{itemize}
        \item[-] V es el volumen en litros
        \item[-] n es la cantidad de sustancia en moles
        \item[-] R es la constante de los gases ideales
        \item[-] T es la temperatura en kelvin
        \item[-] P es la presión en atmósferas
      \end{itemize}
    \end{itemize}
\end{enumerate}

\chapter{}
\section{Uso de pipetas}

\begin{enumerate}
  \item Pipeteo de Agua Destilada:
    \begin{itemize}
      \item Se colocó agua destilada en un vaso de precipitados.
      \item Se utilizó una propipeta manual para pipetear alícuotas de 10 ml y 0.5 ml.
      \item La capacidad de la pipeta utilizada fue de 10 ml para la primera alícuota y de 5 ml para la segunda.
    \end{itemize}
  \item Pipeteo de $KMnO_4$:
\\Se repitió el procedimiento anterior, pero esta vez utilizando KMnO4 en lugar de agua destilada. No hubo cambios ni en el resultado ni en el procedimiento.
  \item Enrasado del Matraz:
\\Para enrasar correctamente el matraz:
    \begin{itemize}
      \item Asegurarse de que el nivel del líquido esté justo en la parte inferior de la marca de enrase.
      \item Evitamos enrasar por debajo o por encima de la marca, ya que esto afectará la precisión de la medición.
    \end{itemize}
  \item Limpieza del Material:
El procedimiento de limpieza incluyó:
    \begin{itemize}
      \item Enjuague inicial de todo el material.
      \item Aplicación de detergente mezclado con agua.
      \item Enjuague final con agua destilada.
    \end{itemize}
\end{enumerate}

\section*{Observaciones}
Durante el procedimiento, se derramaron algunas gotas de 0.5 ml de agua. Esto podría afectar la precisión de las mediciones y debe tenerse en cuenta al analizar los resultados.

\chapter{}
\section{Espectros de emisión}
En esta experiencia se colocaron distintas sales metálicas en una flama.
Para esto se utilizó una cuchara para las sales en estado solido y un rociador para las sales en 
estado acuoso.
Observamos como la llama cambiaba de color dependiendo de la sal introducida.
Las sales utilizadas contenían cationes de $Na^{+} ; Ba^{++} ; Cu^{++} ; Li^{+} ; MG^{++} ; K^{+}$.

Teniendo en cuenta el los espectros de cada uno, se confeccionó la siguiente tabla:

\begin{table}[h!]
\centering
\begin{tabular}{|c|c|c|}
  \hline
  \textbf{} & \textbf{Color} & \textbf{Metal} \\
  \hline
  Muestra 1 & Verde claro & $Ba$ \\
  \hline
  Muestra 2 & Anaranjado & $Na$\\
  \hline
  Muestra 3 & Verde & $Cu$ \\
  \hline
  Muestra 4 & Rojo & $Li$\\
  \hline
  Muestra 5 & Incoloro & $Mg$\\
  \hline
  Muestra 6 & Lila & $K$ \\
  \hline
\end{tabular}
\label{tab:colores_metales}
\end{table}

\section{Cálculos de energía}
Con la ecuación $E=h\cdot\frac{c}{\lambda}$ y aproximando la longitud de onda con la siguiente
tabla, pudimos estimar la energía de los fotones emitidos.

\begin{table}[h!]
  \centering
  \begin{tabular}{|c|c|}
    \hline
    Color & Longitud de onda ($\lambda$)\\
    \hline
    Violeta & $380-450\hspace{1mm}nm$ \\
    \hline
    Azul & $450-495\hspace{1mm} nm$ \\
    \hline
    Verde & $495-570\hspace{1mm} nm$ \\
    \hline
    Amarillo & $570-590\hspace{1mm} nm$ \\
    \hline
    Naranja & $590-620\hspace{1mm} nm$ \\
    \hline
    Rojo & $620-750\hspace{1mm} nm$ \\
    \hline
  \end{tabular}
\end{table}

\begin{minipage}[t]{0.48\textwidth}
  $$
  \begin{aligned}
    E_{M1}& \approx 6,63\times10^{-34}\cdot \frac{3\times10^8 \frac{m}{s} }{ 550 \hspace{1mm}nm}\\
    E_{M1}& \approx 3,616 \times10^{-19}
  \end{aligned}
  $$
  \vspace{5mm}
  $$
  \begin{aligned}
    E_{M2}& \approx 6,63\times10^{-34}\cdot \frac{3\times10^8 \frac{m}{s} }{ 605 \hspace{1mm}nm}\\
    E_{M2}& \approx  3,287 \times10^{-19}
  \end{aligned}
  $$
  \vspace{5mm}
  $$
  \begin{aligned}
    E_{M3}& \approx 6,63\times10^{-34}\cdot \frac{3\times10^8 \frac{m}{s} }{ 532,5 \hspace{1mm}nm}\\
    E_{M3}& \approx  3,735 \times10^{-19}
  \end{aligned}
  $$
\end{minipage}
\hfill
\begin{minipage}[t]{0.48\textwidth}
  $$
  \begin{aligned}
    E_{M4}& \approx 6,63\times10^{-34}\cdot \frac{3\times10^8 \frac{m}{s} }{ 685 \hspace{1mm}nm}\\
    E_{M4}& \approx 2,903 \times10^{-19}
  \end{aligned}
  $$
  \vspace{2mm}

  El espectro de emisión de la Muestra 5 se encuentra fuera del espectro visible, por ende 
  no es posible aproximar su energía por este método.

  \vspace{2mm}
  $$
  \begin{aligned}
    E_{M6}& \approx 6,63\times10^{-34}\cdot \frac{3\times10^8 \frac{m}{s} }{ 415 \hspace{1mm}nm}\\
    E_{M6}& \approx  4,792 \times10^{-19}
  \end{aligned}
  $$
\end{minipage}

El criterio para estimar las distintas longitudes de onda fue tomar el valor intermedio para cada
color en la tabla, a excepción del verde claro, para el cual se tomó un valor un poco mas elevado,
ya que este tiende a un rojo.

\section{Conclusión}

Esta técnica puede ser usada para caracterizar una sustancia simple, ya que cada una de estas tiene
un espectro de emisión característico. Sin embargo, de la forma que fue realizado el experimento en
el laboratorio, es difícil poder afirmar con seguridad que elemento se esta exponiendo a la llama,
ya que distintos espectros pueden ser muy similares a nuestros ojos, e incluso algunos invisibles,
como es el caso de la muestra 5. Pero, con el material apropiado, es posible afirmar con
gran precisión la presencia de un elemento, de hecho, esta técnica es muy utilizada en astronomía,
por ejemplo,para conocer la composición de estrellas u otros cuerpos celestes.

\chapter{}
\section{Determinación del volumen molar}

En este ejercicio se espera poder comprar teórica y prácticamente una reacción en el laboratorio.\\

\subsection{TEÓRICO}

Lo primero que hacemos es plantear la reacción y balancearla:\\
$$
\begin{aligned}
    HCl + Mg &= MgCl_2 + H_2\\[6pt]
    &\Downarrow\\[6pt]
\end{aligned}
$$
$$
\begin{aligned}
    &2HCl + &&Mg \, = &&MgCl_2 + &&H_2 \\[6pt]
    &\downarrow &&\downarrow &&\downarrow &&\downarrow \\
    &72g &&24g &&94g &&2g \\[12pt]
\end{aligned}
$$

Siendo que tenemos $0.365g \, HCl$ y $0.0095g \, Mg$\\

Calculamos cual es el reactivo limitante:\\

$$
\begin{aligned}
    24g \, Mg &\rightarrow&& 94g \, MgCl_2 \\[6pt]
    0.0095g \, Mg &\rightarrow&& 0.00372g \, MgCl_2\\[6pt]
    \\
    24g \, Mg &\rightarrow&& 72g \, HCl \\[6pt]
    0.0095g \, Mg &\rightarrow&& 0.0285g \, HCl \rightarrow \text{Esta claro que el $Mg$ es el reactivo limitante.}\\[6pt]
\end{aligned}
$$

Ahora calculamos cuantos gramos de $H_2$ obtendríamos con $0.0095g \, Mg$ \\

$$
\begin{aligned}
    24g \, Mg &\rightarrow&& 2g \, H_2 \\[6pt]
    0.0095g \, Mg &\rightarrow&& 0.000791g \, H_2\\[6pt]
\end{aligned}
$$

Luego calculamos la cantidad de moles a la que equivale esa masa de hidrógeno.\\

$$
\begin{aligned}
    2g \, H_2  &\rightarrow&& 1 \, \text{Mol} \\[6pt]
    0.000791g \, H_2 &\rightarrow&& 0.000395 \,\text{Mol}  \\[6pt]
\end{aligned}
$$

Obtenemos los datos ambientales de la habitación:\\[6pt]
$P = 748mm \, Hg = 0.9842 \, \text{atm} \hspace{3cm} T= 12 \text{°C} = 285 \text{°K}$\\[6pt]
Procedemos a calcular el volumen teórico que debería ocupar esa cantidad de hidrógeno.\\

$$
\begin{aligned}
    V&=\frac{n . R . T}{P}\\[6pt]
    V&=\frac{0.000395 \, \text{Mol}\, . \, 0.082 \frac{L \, . atm}{\text{Mol} \, . K} \, . \, 285 \text{°K}}{0.9842 \, \text{atm}}\\[6pt]
    V&=9.4 \, mL\\[6pt]
\end{aligned}
$$

\subsection{PRACTICO}

En la practica usamos un eudiómetro para medir el volumen del Gas:\\

Primero calculamos el volumen muerto, el cual es $2.6mL$ para luego medir el volumen que ocupaba el hidrógeno, el cual era de $10.8mL$

\end{document}
