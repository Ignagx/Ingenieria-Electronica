\documentclass[12pt]{report}
\usepackage[left=2.5cm,right=2.5cm,top=3cm,bottom=3cm]{geometry}
\usepackage{fancyhdr}
\usepackage{etoolbox}
\usepackage{titlesec}
\usepackage{titling} % Para personalizar el título
\usepackage{graphicx}
\usepackage{hyperref}
\usepackage{amsmath}
\usepackage{indentfirst}

\geometry{a4paper}

% Configuración de cabecera y pie de página
\pagestyle{fancy}
\fancyhf{} 
\fancyhead[L]{UTN-FRC}
\fancyhead[C]{QUIMICA: TPL1}
\fancyhead[R]{2R3}
\renewcommand{\headrulewidth}{0.4pt}
\fancyfoot[C]{\vfill\thepage}

% Cambio en el estilo de las páginas de capítulo
\patchcmd{\chapter}{\thispagestyle{plain}}{\thispagestyle{fancy}}{}{}

% Tamaños de fuente para matemáticas
\DeclareMathSizes{15}{13}{8}{8}

% Configuración del título del documento
\title{%
  \fontsize{25}{0}\selectfont Universidad Tecnológica Nacional \\
  \fontsize{22}{30}\selectfont Quimica \\
  \fontsize{18}{25}\selectfont TPL1: : Normas de Seguridad, Uso de Material de Vidrio, reactivo limitante y en exceso.
}
\author{
  Jerónimo Billoto - 415323\\
  Luciano Cortesini - 402719\\
  Gaston Grasso - 401892\\
  Ignacio Gil - 401891\\
}
\date{14 / 06 / 2024}

% Formato de títulos y secciones
\titleformat{\chapter}[block]
  {\normalfont\huge\bfseries}{Experiencia \thechapter}{0pt}{\Huge}
\titlespacing*{\chapter}{0pt}{-30pt}{0pt}

\titleformat{\section}[block]
  {\normalfont\Large\bfseries}{\thesection}{2mm}{\Large}
\titlespacing*{\section}{0pt}{3.5ex plus 1ex minus .2ex}{2.3ex plus .2ex}

\titleformat{\subsection}[block]
  {\normalfont\large\bfseries}{}{0pt}{\large}
\titlespacing*{\subsection}{0pt}{3.25ex plus 1ex minus .2ex}{1.5ex plus .2ex}

\begin{document}
\maketitle
\chapter{}
\section{Uso de pipetas}

\chapter{}
\section{Expectros de emisión}
En esta experiencia se colocaron distintas sales metalicas en una flama.
Para esto se utilizó una cuchara para las sales en estado solido y un rociador para las sales en 
estado acuoso.
Observamos como la llama cambiaba de color dependiendo de la sal introducida.
Las sales utilizadas contenian cationes de $Na^{+} ; Ba^{++} ; Cu^{++} ; Li^{+} ; MG^{++} ; K^{+}$.

Teniendo en cuenta el los espectros de cada uno, se confeccionó la siguiente tabla:

\begin{table}[h!]
\centering
\begin{tabular}{|c|c|c|}
  \hline
  \textbf{} & \textbf{Color} & \textbf{Metal} \\
  \hline
  Muestra 1 & Verde claro & $Ba$ \\
  \hline
  Muestra 2 & Anaranjado & $Na$\\
  \hline
  Muestra 3 & Verde & $Cu$ \\
  \hline
  Muestra 4 & Rojo & $Li$\\
  \hline
  Muestra 5 & Incoloro & $Mg$\\
  \hline
  Muestra 6 & Lila & $K$ \\
  \hline
\end{tabular}
\label{tab:colores_metales}
\end{table}

\section{Calculos de energia}
Con la ecuacion $E=h\cdot\frac{c}{\lambda}$ y aproximando la longitud de onda con la siguiente
tabla, pudimos estimar la energia de los fotones emitidos.

\begin{table}[h!]
  \centering
  \begin{tabular}{|c|c|}
    \hline
    Color & Longitud de onda ($\lambda$)\\
    \hline
    Violeta & $380-450\hspace{1mm}nm$ \\
    \hline
    Azul & $450-495\hspace{1mm} nm$ \\
    \hline
    Verde & $495-570\hspace{1mm} nm$ \\
    \hline
    Amarillo & $570-590\hspace{1mm} nm$ \\
    \hline
    Naranja & $590-620\hspace{1mm} nm$ \\
    \hline
    Rojo & $620-750\hspace{1mm} nm$ \\
    \hline
  \end{tabular}
\end{table}

\begin{minipage}[t]{0.48\textwidth}
  $$
  \begin{aligned}
    E_{M1}& \approx 6,63\times10^{-34}\cdot \frac{3\times10^8 \frac{m}{s} }{ 550 \hspace{1mm}nm}\\
    E_{M1}& \approx 3,616 \times10^{-19}
  \end{aligned}
  $$
  \vspace{5mm}
  $$
  \begin{aligned}
    E_{M2}& \approx 6,63\times10^{-34}\cdot \frac{3\times10^8 \frac{m}{s} }{ 605 \hspace{1mm}nm}\\
    E_{M2}& \approx  3,287 \times10^{-19}
  \end{aligned}
  $$
  \vspace{5mm}
  $$
  \begin{aligned}
    E_{M3}& \approx 6,63\times10^{-34}\cdot \frac{3\times10^8 \frac{m}{s} }{ 532,5 \hspace{1mm}nm}\\
    E_{M3}& \approx  3,735 \times10^{-19}
  \end{aligned}
  $$
\end{minipage}
\hfill
\begin{minipage}[t]{0.48\textwidth}
  $$
  \begin{aligned}
    E_{M4}& \approx 6,63\times10^{-34}\cdot \frac{3\times10^8 \frac{m}{s} }{ 685 \hspace{1mm}nm}\\
    E_{M4}& \approx 2,903 \times10^{-19}
  \end{aligned}
  $$
  \vspace{2mm}

  El espectro de emisión de la Muestra 5 se encuentra fuera del espectro visible, por ende 
  no es posible aproximar su energía por este método.

  \vspace{2mm}
  $$
  \begin{aligned}
    E_{M6}& \approx 6,63\times10^{-34}\cdot \frac{3\times10^8 \frac{m}{s} }{ 415 \hspace{1mm}nm}\\
    E_{M6}& \approx  4,792 \times10^{-19}
  \end{aligned}
  $$
\end{minipage}

El criterio para estimar las distintas longitudes de onda fue tomar el valor intermedio para cada
color en la tabla, a excepción del verde claro, para el cual se tomó un valor un poco mas alevado,
ya que este tiende a un rojo.

\section{Conclusión}

Esta técnica puede ser usada para caracterizar una sustancia simple, ya que cada una de estas tiene
un espectro de emisión característico. Sin embargo, de la forma que fué realizado el experimento en
el laboratorio, es dificil poder afirmar con seguridad que elemento se esta exponiendo a la llama,
ya que distintos espectros pueden ser muy similares a nuestros ojos, e incluso algunos invisibles,
como es el caso de la muestra 5. Pero, con el material apropiado, es posible afirmar con
gran presicion la presencia de un elemento, de hecho, esta tecnica es muy utilizada en astronomía,
por ejemplo,para conocer la composición de estrellas u otros cuerpos celestes.

\chapter{}
\section{Determinacion del volumen molar}

En este ejercicio se espera poder comprar teorica y practicamente una reaccion en el laboratorio.\\

\subsection{TEÓRICO}

Lo primero que hacemos es plantear la reaccion y balancearla:\\
$$
\begin{aligned}
    HCl + Mg &= MgCl_2 + H_2\\[6pt]
    &\Downarrow\\[6pt]
\end{aligned}
$$
$$
\begin{aligned}
    &2HCl + &&Mg \, = &&MgCl_2 + &&H_2 \\[6pt]
    &\downarrow &&\downarrow &&\downarrow &&\downarrow \\
    &72g &&24g &&94g &&2g \\[12pt]
\end{aligned}
$$

Siendo que tenemos $0.365g \, HCl$ y $0.0095g \, Mg$\\

Calculamos cual es el reactivo limitante:\\

$$
\begin{aligned}
    24g \, Mg &\rightarrow&& 94g \, MgCl_2 \\[6pt]
    0.0095g \, Mg &\rightarrow&& 0.00372g \, MgCl_2\\[6pt]
    \\
    24g \, Mg &\rightarrow&& 72g \, HCl \\[6pt]
    0.0095g \, Mg &\rightarrow&& 0.0285g \, HCl \rightarrow \text{Esta claro que el $Mg$ es el reactivo limitante.}\\[6pt]
\end{aligned}
$$

Ahora calculamos cuantos gramos de $H_2$ obtendriamos con $0.0095g \, Mg$ \\

$$
\begin{aligned}
    24g \, Mg &\rightarrow&& 2g \, H_2 \\[6pt]
    0.0095g \, Mg &\rightarrow&& 0.000791g \, H_2\\[6pt]
\end{aligned}
$$

Luego calculamos la cantidad de moles a la que equivale esa masa de hidrogeno.\\

$$
\begin{aligned}
    2g \, H_2  &\rightarrow&& 1 \, \text{Mol} \\[6pt]
    0.000791g \, H_2 &\rightarrow&& 0.000395 \,\text{Mol}  \\[6pt]
\end{aligned}
$$

Obtenemos los datos ambientales de la habitacion:\\[6pt]
$P = 748mm \, Hg = 0.9842 \, \text{atm} \hspace{3cm} T= 12 \text{°C} = 285 \text{°K}$\\[6pt]
Procedemos a calcular el volumen teorico que deberia ocupar esa cantidad de hidrogeno.\\

$$
\begin{aligned}
    V&=\frac{n . R . T}{P}\\[6pt]
    V&=\frac{0.000395 \, \text{Mol}\, . \, 0.082 \frac{L \, . atm}{\text{Mol} \, . K} \, . \, 285 \text{°K}}{0.9842 \, \text{atm}}\\[6pt]
    V&=9.4 \, mL\\[6pt]
\end{aligned}
$$

\subsection{PRACTICO}

En la practica usamos un eudiometro para medir el volumen del Gas:\\

Primero calculamos el volumen muerto, el cual es $2.6mL$ para luego medir el volumen que ocupaba el hidrogeno, el cual era de $10.8mL$

\end{document}
