\documentclass[12pt]{report}
\usepackage[left=2.5cm,right=2.5cm,top=3cm,bottom=3cm]{geometry}
\usepackage{fancyhdr}
\usepackage{etoolbox}
\usepackage{titlesec}
\usepackage{titling} % Para personalizar el título
\usepackage{graphicx}
\usepackage{hyperref}
\usepackage{amsmath}
\usepackage{indentfirst}

\geometry{a4paper}

% Configuración de cabecera y pie de página
\pagestyle{fancy}
\fancyhf{} 
\fancyhead[L]{UTN-FRC}
\fancyhead[C]{QUIMICA: TPL1}
\fancyhead[R]{2R3}
\renewcommand{\headrulewidth}{0.4pt}
\fancyfoot[C]{\vfill\thepage}

% Cambio en el estilo de las páginas de capítulo
\patchcmd{\chapter}{\thispagestyle{plain}}{\thispagestyle{fancy}}{}{}

% Tamaños de fuente para matemáticas
\DeclareMathSizes{15}{13}{8}{8}

% Configuración del título del documento
\title{%
  \fontsize{25}{0}\selectfont Universidad Tecnológica Nacional \\
  \fontsize{22}{30}\selectfont Quimica \\
  \fontsize{18}{25}\selectfont TPL1: : Normas de Seguridad, Uso de Material de Vidrio, reactivo limitante y en exceso.
}
\author{
  Jerónimo Billoto - 415323\\
  Luciano Cortesini - 402719\\
  Gaston Grasso - 401892\\
  Ignacio Gil - 401891\\
}
\date{14 / 06 / 2024}

% Formato de títulos y secciones
\titleformat{\chapter}[block]
  {\normalfont\huge\bfseries}{Experiencia \thechapter}{0pt}{\Huge}
\titlespacing*{\chapter}{0pt}{-30pt}{0pt}

\titleformat{\section}[block]
  {\normalfont\Large\bfseries}{\thesection}{2mm}{\Large}
\titlespacing*{\section}{0pt}{3.5ex plus 1ex minus .2ex}{2.3ex plus .2ex}

\titleformat{\subsection}[block]
  {\normalfont\large\bfseries}{}{0pt}{\large}
\titlespacing*{\subsection}{0pt}{3.25ex plus 1ex minus .2ex}{1.5ex plus .2ex}

\begin{document}
\maketitle
\chapter{}
\section{Uso de pipetas}

\chapter{}
\section{Expectros de emisión}
En esta experiencia se colocaron distintas sales metalicas en una flama.
Para esto se utilizó una cuchara para las sales en estado solido y un rociador para las sales en 
estado acuoso.
Observamos como la llama cambiaba de color dependiendo de la sal introducida.
Las sales utilizadas contenian cationes de $Na^{+} ; Ba^{++} ; Cu^{++} ; Li^{+} ; MG^{++} ; K^{+}$.

Teniendo en cuenta el los espectros de cada uno, se confeccionó la siguiente tabla:

\begin{table}[h!]
\centering
\begin{tabular}{|c|c|c|}
  \hline
  \textbf{} & \textbf{Color} & \textbf{Metal} \\
  \hline
  Muestra 1 & Verde claro & $Ba$ \\
  \hline
  Muestra 2 & Anaranjado & $Na$\\
  \hline
  Muestra 3 & Verde & $Cu$ \\
  \hline
  Muestra 4 & Rojo & $Li$\\
  \hline
  Muestra 5 & Incoloro & $Mg$\\
  \hline
  Muestra 6 & Lila & $K$ \\
  \hline
\end{tabular}
\caption{Tabla de colores y metales}
\label{tab:colores_metales}
\end{table}

\section{Calculos de energia}
Con la ecuacion $E=h\cdot\frac{c}{\lambda}$ y aproximando la longitud de onda con la tabla a continuación,
pudimos estimar la energia de los fotones emitidos.

\begin{table}[h!]
  \centering
  \begin{tabular}{|c|c|}
    \hline
    Color & Longitud de onda ($\lambda$)\\
    \hline
    Violeta & $380-450\hspace{1mm}nm$ \\
    \hline
    Azul & $450-495\hspace{1mm} nm$ \\
    \hline
    Verde & $495-570\hspace{1mm} nm$ \\
    \hline
    Amarillo & $570-590\hspace{1mm} nm$ \\
    \hline
    Naranja & $590-620\hspace{1mm} nm$ \\
    \hline
    Rojo & $620-750\hspace{1mm} nm$ \\
    \hline
  \end{tabular}
\end{table}

\begin{minipage}[t]{0.48\textwidth}
  $$
  \begin{aligned}
    E_{M1}&=6,63\times10^{-34}\cdot \frac{3\times10^8 \frac{m}{s} }{532,5\hspace{1mm}nm}\\
    E_{M1}&=3,862\times10^{-19}
  \end{aligned}
  $$
  \vspace{5mm}
  $$
  \begin{aligned}
    E_{M2}&=6,63\times10^{-34}\cdot \frac{3\times10^8 \frac{m}{s} }{ 605 \hspace{1mm}nm}\\
    E_{M2}&= 3,287 \times10^{-19}
  \end{aligned}
  $$
\end{minipage}
\hfill
\begin{minipage}[t]{0.48\textwidth}
  $$asdlkasdlhja$$
\end{minipage}

\chapter{}
\section{Determinacion del volumen molar}

En este ejercicio se espera poder comprar teorica y practicamente una reaccion en el laboratorio.\\

\subsection{TEÓRICO}

Lo primero que hacemos es plantear la reaccion y balancearla:\\
$$
\begin{aligned}
    HCl + Mg &= MgCl_2 + H_2\\[6pt]
    &\Downarrow\\[6pt]
\end{aligned}
$$
$$
\begin{aligned}
    &2HCl + &&Mg \, = &&MgCl_2 + &&H_2 \\[6pt]
    &\downarrow &&\downarrow &&\downarrow &&\downarrow \\
    &72g &&24g &&94g &&2g \\[12pt]
\end{aligned}
$$

Siendo que tenemos $0.365g \, HCl$ y $0.0095g \, Mg$\\

Calculamos cual es el reactivo limitante:\\

$$
\begin{aligned}
    24g \, Mg &\rightarrow&& 94g \, MgCl_2 \\[6pt]
    0.0095g \, Mg &\rightarrow&& 0.00372g \, MgCl_2\\[6pt]
    \\
    24g \, Mg &\rightarrow&& 72g \, HCl \\[6pt]
    0.0095g \, Mg &\rightarrow&& 0.0285g \, HCl \rightarrow \text{Esta claro que el $Mg$ es el reactivo limitante.}\\[6pt]
\end{aligned}
$$

Ahora calculamos cuantos gramos de $H_2$ obtendriamos con $0.0095g \, Mg$ \\

$$
\begin{aligned}
    24g \, Mg &\rightarrow&& 2g \, H_2 \\[6pt]
    0.0095g \, Mg &\rightarrow&& 0.000791g \, H_2\\[6pt]
\end{aligned}
$$

Luego calculamos la cantidad de moles a la que equivale esa masa de hidrogeno.\\

$$
\begin{aligned}
    2g \, H_2  &\rightarrow&& 1 \, \text{Mol} \\[6pt]
    0.000791g \, H_2 &\rightarrow&& 0.000395 \,\text{Mol}  \\[6pt]
\end{aligned}
$$

Obtenemos los datos ambientales de la habitacion:\\[6pt]
$P = 748mm \, Hg = 0.9842 \, \text{atm} \hspace{3cm} T= 12 \text{°C} = 285 \text{°K}$\\[6pt]
Procedemos a calcular el volumen teorico que deberia ocupar esa cantidad de hidrogeno.\\

$$
\begin{aligned}
    V&=\frac{n . R . T}{P}\\[6pt]
    V&=\frac{0.000395 \, \text{Mol}\, . \, 0.082 \frac{L \, . atm}{\text{Mol} \, . K} \, . \, 285 \text{°K}}{0.9842 \, \text{atm}}\\[6pt]
    V&=9.4 \, mL\\[6pt]
\end{aligned}
$$

\subsection{PRACTICO}

En la practica usamos un eudiometro para medir el volumen del Gas:\\

Primero calculamos el volumen muerto, el cual es $2.6mL$ para luego medir el volumen que ocupaba el hidrogeno, el cual era de $10.8mL$

\end{document}
