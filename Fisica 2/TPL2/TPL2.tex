\documentclass[12pt]{report}
\usepackage[left=2.5cm,right=2.5cm,top=3cm,bottom=3cm]{geometry}
\usepackage{fancyhdr}
\usepackage{etoolbox}
\usepackage{titlesec}
\usepackage{titling} % Para personalizar el título
\usepackage{graphicx}
\usepackage{hyperref}
\usepackage{amsmath}

\geometry{a4paper}

% Configuración de cabecera y pie de página
\pagestyle{fancy}
\fancyhf{} 
\fancyhead[L]{UTN-FRC}
\fancyhead[C]{FÍSICA 2: TPL1}
\fancyhead[R]{2R3}
\renewcommand{\headrulewidth}{0.4pt}
\fancyfoot[C]{\vfill\thepage}

% Cambio en el estilo de las páginas de capítulo
\patchcmd{\chapter}{\thispagestyle{plain}}{\thispagestyle{fancy}}{}{}

% Tamaños de fuente para matemáticas
\DeclareMathSizes{12}{13}{6}{5}

% Configuración del título del documento
\title{%
  \fontsize{25}{0}\selectfont Universidad Tecnológica Nacional \\
  \fontsize{22}{30}\selectfont Física 2 \\
  \fontsize{18}{25}\selectfont TPL2: Capacitores
}
\author{
  Franco Palombo\\
  Gaston Grasso\\
  Ignacio Gil\\
  Santino Noccetti\\
}
\date{05 / 06 / 2024}

% Formato de títulos y secciones
\titleformat{\chapter}[block]
  {\normalfont\huge\bfseries}{}{0pt}{\Huge}
\titlespacing*{\chapter}{0pt}{-30pt}{20pt}

\titleformat{\section}[block]
  {\normalfont\Large\bfseries}{}{0pt}{\Large}
\titlespacing*{\section}{0pt}{3.5ex plus 1ex minus .2ex}{2.3ex plus .2ex}

\titleformat{\subsection}[block]
  {\normalfont\large\bfseries}{}{0pt}{\large}
\titlespacing*{\subsection}{0pt}{3.25ex plus 1ex minus .2ex}{1.5ex plus .2ex}

\begin{document}
\maketitle

\section{Introducción 1: El Capacitor Plano}

¡Empecemos! Ingresá al simulador que te adjunté. Si tenés algún problema con el archivo, podés descargarlo de forma directa a partir del link incorporado en la siguiente imagen.

Tildá las casillas “Capacidad” y “Carga de la placa”. Mantené la pila conectada e intenta responder a las siguientes preguntas:

\begin{enumerate}
    \item Al aumentar la diferencia de potencial entre las placas, ¿Qué sucede con la carga de la placa positiva del capacitor? ¿y con la capacidad?
    \item Fijá la diferencia de potencial a 1,5 V, habilitá el voltímetro y explica qué sucede con la carga, con la diferencia de potencial y con la capacidad, al variar el área de las placas.
    \item Modificá el espacio entre las placas y comenta qué sucede con la carga, la capacidad y la diferencia de potencial.
\end{enumerate}

\section{Introducción 2: El Campo Eléctrico}

Habilitá el detector de campo eléctrico, y manteniendo la pila conectada responde a las siguientes preguntas:

\begin{enumerate}
    \item ¿Qué sucede con la intensidad del campo eléctrico al variar el área de las placas? ¿Podrías explicar por qué?
    \item ¿Qué sucede con la intensidad del campo al variar el espaciamiento entre las placas? ¿A qué se debe?
\end{enumerate}

\section{Desconexión de la Fuente}

\begin{enumerate}
    \item Fijá la tensión de la fuente a 1,5 V, la separación de las placas en 10 mm y el área a 100 mm$^2$. Mantené el voltímetro y el sensor de campo eléctrico conectados y luego desconectá la fuente.
    \begin{enumerate}
        \item ¿Qué sucede con la carga, diferencia de potencial, capacidad e intensidad del campo eléctrico al disminuir el espaciamiento entre las placas?
        \\
        3Foto1
        
        Al disminuir la distancia entre las placas, la diferencia de potencial del capacitor disminuye, mientras que la capacidad del mismo aumenta. En cambio, la carga  y el campo eléctrico permanecen constantes.
        
        3Foto2
        \\
        \item ¿Qué sucede con la carga, diferencia de potencial, capacidad e intensidad de campo eléctrico al aumentar el área entre las placas?
        \\
        3Foto1

        Al aumentar el área de las placas, la diferencia de potencial disminuye considerablemente, al igual que el campo entre las placas, mientras que la capacidad aumenta considerablemente. La carga permanece constante.

        3Foto3
        \\
    \end{enumerate}
    \item Conecta nuevamente el capacitor a la fuente de 1,5 V, fijá la separación de las placas a 5 mm, y el área a 400 mm$^2$. Ahora desconecta el capacitor de la fuente y responde:
    \begin{enumerate}
        \item ¿Qué sucede con la carga, diferencia de potencial, capacidad e intensidad de campo eléctrico al aumentar el espaciamiento entre las placas?
        \\
        3Foto4

        Al aumentar la distancia entre las placas, la diferencia de potencial del capacitor aumenta, mientras que la capacidad del mismo disminuye. En cambio, la carga  y el campo eléctrico permanecen constantes.

        3Foto5
        \\
        \item ¿Qué sucede con la carga, diferencia de potencial, capacidad e intensidad de campo eléctrico al disminuir el área entre las placas?
        \\
        3Foto4

        Al disminuir el área de las placas, la diferencia de potencial aumenta considerablemente, al igual que el campo entre las placas, mientras que la capacidad disminuye considerablemente. La carga permanece constante.

        3Foto6
        \\
    \end{enumerate}
    \item ¿Podrías explicar lo sucedido en 1 y en 2?
    \\
    Para todos los casos, la carga permanece constante debido a que el capacitor, permanece "en vacío" para todas las pruebas. Esto implica que no hay movimiento de cargas en ningún momento.

    Para todos los casos, la variación de capacidad está dictada por la fórmula de capacitancia:

\[C=\frac{Q}{V_{ab}}\]

    Debido a que la carga del capacitor permanece siempre constante, el único parámetro que puede alterar la capacidad es la diferencia de potencial, que, como se vio antes, varía en todos los casos.
    
    La diferencia de potencial, se puede definir como:
    
\[V_a-V_b=\int_{a}^{b}\vec{E}\cdot d \vec{l}\]

    Cuando variamos la distancia entre las placas, el campo eléctrico permanece constante, por lo que la variación está dictada por \(d \vec{l}\). La variación de la distancia es directamente proporcional a la diferencia de potencial entre las placas. Mientras más chica sea la distancia, menor el voltaje, y mientras más grande, mayor el voltaje.

    Ahora, para los casos en los que el campo eléctrico varía, la distancia entre las placas permanece constante. Podemos definir el campo eléctrico como:
    
\[\vec{E}=\frac{\sigma}{\epsilon_0}=\frac{Q}{\epsilon_0 A}\]

    El aumento en el área tiene un efecto inverso en el campo eléctrico. Esto es porque al mantenerse las cargas constantes, si aumentamos el área disminuye drásticamente la densidad de carga por unidad de área. En cambio, si se disminuye el área, las cargas son las mismas pero se tienen que acomodar en una superficie más chica que antes, por lo que la densidad de carga aumenta drásticamente.

\end{enumerate}

\section{Introduzcamos un Dieléctrico}

\begin{enumerate}
    \item Seleccioná el aislante dieléctrico que prefieras y, con la fuente conectada, introdúcelo entre las placas del capacitor.
        \\\\Se selecciono como aislante al teflon, que tiene una constante dielectrica $k = 2,1$
    \begin{enumerate}
        \item ¿Qué cambios se observan en el material? (Puedes tildar la casilla “mostrar cargas en exceso” para ver mejor el fenómeno).
        \\\\Al introducir el dielectrico entre las placas, este se polariza. Esto se puede visualizar mostrando las cargas en exceso al ver que las cargas opuestas dentro del dielectrico se trasladan hacia los extremos del material, atrayendose las cargas positivas del dielectrico a las negativas de la placa y las cargas negativas del dielectrico a las positivas de la placa.
        \item ¿Cuánto vale el campo eléctrico resultante entre las placas? ¿Cuánto valía el campo eléctrico entre las placas antes de introducir el dieléctrico? ¿Qué puedes decir al respecto?
        \\\\El campo electrico resultante entre las placas vale $150\frac{V}{m}$. Antes de introducir el dielectrico, tambien valia $150\frac{V}{m}$. Esto se debe a que en el caso sin dielectrico, el unico campo que se produce es el que hay entre las placas y que es proporcional a la diferencia de potencial y al area de las placas; en el caso con dielectrico entre las placas, el campo entre las placas aumenta a $E_p = 315\frac{V}{m}$ pero hay que tener en cuenta tambien el campo inducido que se le resta, que vale $E_i = 165\frac{V}{m}$, siendo $E_p - E_i = 150\frac{V}{m}$.
        \item La simulación te presenta el valor del campo eléctrico inducido, el campo eléctrico en el vacío, y el campo eléctrico resultante. En base a éstos, intenta calcular la constante dieléctrica y verificar el valor proporcionado por el simulador.
        \item ¿Qué sucede con las cargas en las placas metálicas, la diferencia de potencial, la capacidad cuando se introduce el dieléctrico?
        \item Habilita la casilla de “Energía almacenada” y cuéntame qué ocurre con la energía al introducir un dieléctrico. Realiza el cociente entre la energía almacenada con dieléctrico y la almacenada sin dieléctrico. Compara dicho cociente con la constante dieléctrica.
    \end{enumerate}
    \item Ahora, retira el dieléctrico y asegúrate que la pila aporte una diferencia de potencial de 1,5 V. Luego, desconecta la fuente y vuelve a introducir el dieléctrico.
    \begin{enumerate}
        \item ¿Cuánto vale el campo eléctrico resultante entre las placas? ¿Y entre las placas cuando no hay material? ¿Qué puedes decir al respecto?
        \item Vuelve a verificar el valor de la constante dieléctrica en base a la intensidad de los campos. ¿Se obtiene el mismo valor que el obtenido con la fuente conectada?
        \item ¿Qué sucede con las cargas en las placas metálicas, la diferencia de potencial, la capacidad cuando se introduce el dieléctrico?
        \item ¿Qué ocurre con la energía al introducir un dieléctrico? Realiza el cociente entre la energía almacenada con dieléctrico y la almacenada sin dieléctrico. Compara dicho cociente con la constante dieléctrica.
    \end{enumerate}
\end{enumerate}

\section{Conexión de Capacitores}

\begin{enumerate}
    \item Arma un circuito serie de dos capacitores de capacidades $C_1 = 2 \times 10^{-13} \, \text{F}$ y $C_2 = 3 \times 10^{-13} \, \text{F}$ con una batería de 1,5 V.
    \begin{enumerate}
        \item Mide los campos eléctricos de ambos capacitores. ¿Qué relación guardan entre sí?
        \item Mide la carga de cada capacitor. ¿Qué relación guardan entre sí?
        \item Mide las diferencias de potencial entre las placas de cada capacitor. ¿Qué relación guardan con la diferencia de potencial aportada por la batería?
        \item Calcula la capacidad de un supuesto capacitor que acumule la misma carga, y esté sometido a la misma diferencia de potencial total que el conjunto $\{C_1, C_2\}$ (capacitor equivalente).
        \item Verifica el cálculo anterior reemplazando ambos capacitores por el equivalente y verificando su carga, campo eléctrico y diferencia de potencial.
    \end{enumerate}
    
    \item Arma un circuito paralelo de dos capacitores de capacidades $C_1 = 1 \times 10^{-13} \, \text{F}$ y $C_2 = 2 \times 10^{-13} \, \text{F}$ con una batería de 1,5 V.
    \begin{enumerate}
        \item Mide los campos eléctricos de ambos capacitores. ¿Qué relación guardan con las capacidades?
        \item Mide la carga de cada capacitor. ¿Qué relación guardan con los campos eléctricos?
        \item Mide las diferencias de potencial entre las placas de cada capacitor. ¿Qué relación guardan con la diferencia de potencial aportada por la batería?
        \item Calcula la capacidad de un capacitor que, sometido a la misma diferencia de potencial, acumule la misma carga total que el conjunto $\{C_1, C_2\}$.
        \item Verifica el cálculo anterior reemplazando ambos capacitores por el equivalente y verificando su carga, campo eléctrico y diferencia de potencial.
    \end{enumerate}
\end{enumerate}

\section{Circuitos Mixtos}

Construye dos circuitos mixtos como los de las siguientes figuras. Luego, mide las diferencias de potencial a los bornes de cada capacitor, el campo eléctrico entre las placas de cada uno, la carga de cada uno y la energía total acumulada. Luego, verifica todas estas cantidades de manera analítica.

\end{document}

