\documentclass[12pt]{report}

\title{Universidad Tecnológica Nacional\\TPL1: Calorimetria}
\author{
Luciano Cortesini\\
Franco Palombo\\
Gaston Grasso\\
Ignacio Gil\\
Santino Noccetti\\
Veronica Sticotti
}
\date{Abril 2024}

\begin{document}
\maketitle

\chapter{Experiencia 1}
\section{Objetivo}
En esta primera experiencia se busca medir las perdidas de calor de un calorímetro (equivalente en agua: $\pi$) mediante el método indirecto. 

\section{Procedimiento}
Para esto se ingresó una masa de $92 g$ de agua a una temperatura de $25^\circ C$.
Luego se agregó al calorímetro una porción de agua hirviendo, obteniendo finalmente una temperatura de  $45^\circ C$. 
Para obtener la masa de agua hirviendo que fue ingresada al calorímetro, se vertió todo el contenido en el vaso de precipitado y se pesó la masa total, deduciendo así una masa de agua caliente de $43g$.
\subsection{Calculo de $\pi$}
\begin{table}[htbp]
    \centering
    \begin{tabular}{|c|c|c|c|}
    \hline
    & Agua fria & Agua caliente & Calorimetro\\
    \hline
    masa & $92g$ & $43g$ & $\pi$ \\
    \hline
   $T_i$ & $25^\circ C$ & $100^\circ C$ & $25^\circ C$\\
    \hline
     $T_f$ & \multicolumn{3}{|c|}{$45^\circ C$}\\
    \hline
    \end{tabular}
    \caption{Datos experiencia 1.}
    \label{tab:datos experiencia 1}
\end{table}


\chapter{Experiencia 2}
\section{Objetivo}
En la segunda experiencia, el objetivo era calcular el calor especifico de un metal desconocido.

\section{Procedimiento}
Primero se preparaban $150 g$ de agua a temperatura ambiente y un metal desconocido a $100^\circ C$ que tenia una masa de $229 g$. Se toma la temperatura del agua, se introduce el metal y se cierra la tapa del calorimetro. Finalmente, se mide nuevamente la temperatura de equilibrio.
Estos valores se pueden visualizar en el cuadro \ref{table:data}:

\begin{table}[h!]
\centering
\begin{tabular}{||c c c c||}
    \hline
    Valores & Agua Fria & Calorimetro & Metal Desconocido \\ [0.5ex]
    \hline\hline
    Masa & 144 & 229 & 26.25 \\
    $T_o [^\circ C]$ & 35 & 35 & 35 \\
    \hline
\end{tabular}
\caption{Cuadro de valores medidos}
\label{table:data}
\end{table}




\end{document}


