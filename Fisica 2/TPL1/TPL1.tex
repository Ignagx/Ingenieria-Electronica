\documentclass[12pt]{report}
\usepackage[left=3cm,right=3cm,top=2cm,bottom=2cm]{geometry}

\renewcommand{\chaptername}{Experiencia}

\DeclareMathSizes{12}{13}{6}{5}

\title{Universidad Tecnológica Nacional\\TPL1: Calorimetria}
\author{
Luciano Cortesini\\
Franco Palombo\\
Gaston Grasso\\
Ignacio Gil\\
Santino Noccetti\\
Veronica Sticotti
}
\date{Abril 2024}

\begin{document}
\maketitle

\chapter{}
\section{Objetivo}
En esta primera experiencia se busca medir las perdidas de calor de un calorímetro (equivalente en agua: $\pi$) mediante el método indirecto. 

\section{Procedimiento}
Para esto se ingresó una masa de $92 g$ de agua a una temperatura de $25^\circ C$.
Luego se agregó al calorímetro una porción de agua hirviendo, obteniendo finalmente una temperatura de  $45^\circ C$. 
Para obtener la masa de agua hirviendo que fue ingresada al calorímetro, se vertió todo el contenido en el vaso de precipitado y se pesó la masa total, deduciendo así una masa de agua caliente de $43g$.
\begin{table}[htbp]
    \centering
    \begin{tabular}{|c|c|c|c|}
    \hline
    & Agua Fría (AF) & Agua Caliente (AC) & Calorímetro (C)\\
    \hline
    $Masa[g]$ & 92 & 43 & $\pi$ \\
    \hline
   $T_o[^\circ C]$ & 25 & 100 & 25\\
    \hline
     $T_f[^\circ C]$ & \multicolumn{3}{|c|}{45}\\
    \hline
    \end{tabular}
    \caption{Cuadro de valores medidos en la experiencia 1}
    \label{tab:datos experiencia 1}
\end{table}
\subsection{Calculo de $\pi$}
Para calcular $\pi$ se parte de la idea de que el calorímetro es un sistema cerrado, por ende no intercambia calor con el entorno. por lo que es correcto afirmar que:
$$\sum Q_{Sistema}= 0$$
$$Q_{AF} + Q_{AC} + Q_{C} = 0$$
$$m_{AF} \cdot c_{Agua} \cdot {\Delta t}_{AF} + m_{AC} \cdot c_{Agua} \cdot {\Delta t}_{AF} + \pi \cdot c_{Agua} \cdot {\Delta t}_{C} = 0$$
\hspace{1cm} Sustituimos por cada respectiva formula de calor.
$$c_{Agua} \cdot (m_{AF}  \cdot {\Delta t}_{AF} + m_{AC} \cdot {\Delta t}_{AC} + \pi \cdot {\Delta t}_C) = 0$$
$$m_{AF} \cdot  {\Delta t}_{AF} + m_{AC} \cdot  {\Delta t}_{AC} + \pi \cdot {\Delta t}_C= 0$$
\hspace{1cm} y finalmente despejamos $\pi$.
$$\pi =-\frac{m_{AF} \cdot  {\Delta t}_{AF} + m_{AC} \cdot  {\Delta t}_{AC}}{{\Delta t}_C}$$
$$\pi =-\frac{92g \cdot (45^\circ C-25^\circ C)+ 43g \cdot (45^\circ C-100^\circ C)}{45^\circ C-25^\circ C}$$
$$\pi = 26,25g$$
\chapter{}
\section{Objetivo}
En la segunda experiencia, el objetivo era calcular el calor especifico de un metal desconocido.

\section{Procedimiento}
Primero se preparaban $150 g$ de agua a temperatura ambiente y un metal desconocido a $100^\circ C$ que tenia una masa de $229 g$. Se toma la temperatura del agua, se introduce el metal y se cierra la tapa del calorimetro. Finalmente, se mide nuevamente la temperatura de equilibrio.
Estos valores se pueden visualizar en el cuadro \ref{table:data}:

\begin{table}[h!]
\centering
\begin{tabular}{|c|c|c|c|}
    \hline
    Valores & Agua Fria (AF) & Calorimetro (C) & Metal Desconocido (MD) \\
    \hline
    Masa$[g]$ & 149 & 229 & 26.25 \\
    \hline
    $T_o [^\circ C]$ & 27 & 100 & 27 \\
    \hline
    $T_f [^\circ C]$ & \multicolumn{3}{|c|}{35}\\
    \hline
\end{tabular}
\caption{Cuadro de valores medidos en la experiencia 2}
\label{table:data}
\end{table}

\subsection{Calculo del Calor Específico}

Para el calculo del calor específico utilizamos la formula general de calor que dice:

$$\sum Q_{Sistema} = 0$$

\noindent
Teniendo en cuenta el calor de cada uno de los materiales que utilizamos, nos queda:

$$Q_{AF} + Q_c + Q_{MD} = 0$$
$$m_{AF} \cdot C_{Agua} \cdot {\Delta T}_{AF} + m_C \cdot C_C \cdot {\Delta T}_c + m_{MD} \cdot C_{MD} \cdot {\Delta T}_{MD} = 0$$
$$149g \cdot 1 \frac{cal}{g\cdot^\circ C} \cdot (35^\circ C - 27^\circ C) + 26,25g \cdot 1 \frac{cal}{g\cdot^\circ C} \cdot (35^\circ C - 25^\circ C) + 229g \cdot  C_{MD} \cdot (35^\circ C - 100^\circ C) = 0$$
$$149g \cdot 1 \frac{cal}{g\cdot^\circ C} \cdot 8^\circ C + 26,25g \cdot 1 \frac{cal}{g\cdot^\circ C} \cdot 8^\circ C + 229g \cdot C_{MD}  \cdot (-65)^\circ C = 0$$
$$1.192 cal + 210 cal - 14.885 g\cdot^\circ C \cdot Ce_{MD}  = 0$$

\noindent
Si despejamos $Ce_{MD}$ quedaria:

$$C_{MD}  = \frac{210 cal - 1.192 cal}{-14.885 g\cdot^\circ C}\\$$
$$C_{MD}  = 0.094 \frac{cal}{g\cdot^\circ C}$$

\noindent
Este calor específico nos indica que el metal desconocido era de bronce.

\end{document}
