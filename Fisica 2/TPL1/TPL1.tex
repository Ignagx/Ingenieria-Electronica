\documentclass[12pt]{report}

\title{Universidad Tecnológica Nacional\\TPL1: Calorimetria}
\author{
Luciano Cortesini\\
Franco Palombo\\
Gaston Grasso\\
Ignacio Gil\\
Santino Noccetti\\
Veronica Sticotti
}
\date{Abril 2024}

\begin{document}

\maketitle

\section{Experiencia 1}
En esta primera experiencia se busca medir las perdidas de calor de un calorímetro (equivalente en agua: $\pi$) mediante el método indirecto. 
Para esto se ingresó una masa de $92g$ de agua a una temperatura de $25^\circ C$.
Luego se agregó al calorímetro una porción de agua hirviendo, obteniendo finalmente una temperatura de  $45^\circ C$. 
Para obtener la masa de agua hirviendo que fue ingresada al calorímetro, se vertió todo el contenido en el vaso de precipitado y se pesó la masa total, deduciendo así una masa de agua caliente de $43g$.

\end{document}
