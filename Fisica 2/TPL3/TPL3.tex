\documentclass[12pt]{report}
\usepackage[left=2.5cm,right=2.5cm,top=3cm,bottom=3cm]{geometry}
\usepackage{fancyhdr}
\usepackage{etoolbox}
\usepackage{titlesec}
\usepackage{titling} % Para personalizar el título
\usepackage{graphicx}
\usepackage{hyperref}
\usepackage{amsmath}
\usepackage{circuitikz} % Para dibujar circuitos

\geometry{a4paper}

% Configuración de cabecera y pie de página
\pagestyle{fancy}
\fancyhf{} 
\fancyhead[L]{UTN-FRC}
\fancyhead[C]{FÍSICA 2: TPL3}
\fancyhead[R]{2R3}
\renewcommand{\headrulewidth}{0.4pt}
\fancyfoot[C]{\vfill\thepage}

% Cambio en el estilo de las páginas de capítulo
\patchcmd{\chapter}{\thispagestyle{plain}}{\thispagestyle{fancy}}{}{}

% Tamaños de fuente para matemáticas
\DeclareMathSizes{12}{13}{6}{5}

% Configuración del título del documento
\title{%
  \fontsize{25}{30}\selectfont Universidad Tecnológica Nacional \\
  \fontsize{22}{30}\selectfont Física 2 \\
  \fontsize{18}{25}\selectfont TPL3: Resistividad
}
\author{
  Franco Palombo\\
  Gaston Grasso\\
  Ignacio Gil\\
  Santino Noccetti\\
}
\date{05 / 06 / 2024}

% Formato de títulos y secciones
\titleformat{\chapter}[block]
  {\normalfont\huge\bfseries}{}{0pt}{\Huge}
\titlespacing*{\chapter}{0pt}{-30pt}{20pt}

\titleformat{\section}[block]
  {\normalfont\Large\bfseries}{}{0pt}{\Large}
\titlespacing*{\section}{0pt}{3.5ex plus 1ex minus .2ex}{2.3ex plus .2ex}

\titleformat{\subsection}[block]
  {\normalfont\large\bfseries}{}{0pt}{\large}
\titlespacing*{\subsection}{0pt}{3.25ex plus 1ex minus .2ex}{1.5ex plus .2ex}

\begin{document}
\maketitle
\centering
\begin{circuitikz}
    % Dibujar la fuente de voltaje
    \draw (0,0) to[battery1, l=$V$,invert] (0,4)
    % Dibujar las resistencias en serie
    to[R, l=$R_1$] (3,4)
    to[R, l=$R_2$] (6,4)
    to[R, l=$R_3$] (9,4)
    % Conectar la última resistencia a tierra y cerrar el circuito
    -- (9,0) -- (0,0);
\end{circuitikz}

\vspace{2cm}

\newcommand {\ResUno} {R_1}
\newcommand {\ResDos} {550\Omega}
\newcommand {\ResTres} {R_3}

\centering
\begin{circuitikz}
    % Dibujar la fuente de voltaje
  \draw (0,0) to[battery1, l=$V$, invert] (0,4) -- (2,4) % Dibujar las resistencias en serie 
    to[R, l=$\ResUno$] (2,0) -- (0,0);
  \draw (2,4) -- (4,4)
    to[R, l=$\ResDos$] (4,0) -- (0,0);
  \draw (4,4) -- (6,4)
    to[R, l=$\ResTres$] (6,0) -- (0,0);
\end{circuitikz}

$$R_T  = \left( \frac{1}{\ResUno}+\frac{1}{\ResDos}+\frac{1}{\ResTres} \right) ^{-1}$$

\end{document}

