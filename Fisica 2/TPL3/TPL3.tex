\documentclass[12pt]{report}
\usepackage[left=2.5cm,right=2.5cm,top=3cm,bottom=3cm]{geometry}
\usepackage{fancyhdr}
\usepackage{etoolbox}
\usepackage{titlesec}
\usepackage{titling} % Para personalizar el título
\usepackage{graphicx}
\usepackage{hyperref}
\usepackage{amsmath}
\usepackage{amssymb}
\usepackage{circuitikz} % Para dibujar circuitos
\usepackage{indentfirst}
\usepackage{relsize}

\geometry{a4paper}

% Configuración de cabecera y pie de página
\pagestyle{fancy}
\fancyhf{} 
\fancyhead[L]{UTN-FRC}
\fancyhead[C]{FÍSICA 2: TPL3}
\fancyhead[R]{2R3}
\renewcommand{\headrulewidth}{0.4pt}
\fancyfoot[C]{\vfill\thepage}

% Cambio en el estilo de las páginas de capítulo
\patchcmd{\chapter}{\thispagestyle{plain}}{\thispagestyle{fancy}}{}{}

% Tamaños de fuente para matemáticas
\DeclareMathSizes{12}{13}{8}{8}
\newcommand {\LEpsilon}{\mathlarger\varepsilon}

% Configuración del título del documento
\title{%
  \fontsize{25}{30}\selectfont Universidad Tecnológica Nacional \\
  \fontsize{22}{30}\selectfont Física 2 \\
  \fontsize{18}{25}\selectfont TPL3: Resistividad
}
\author{
  Franco Palombo\\
  Gaston Grasso\\
  Ignacio Gil\\
  Santino Noccetti\\
}
\date{05 / 06 / 2024}

% Formato de títulos y secciones
\titleformat{\chapter}[block]
  {\normalfont\huge\bfseries}{}{0pt}{\Huge}
\titlespacing*{\chapter}{0pt}{-30pt}{20pt}

\titleformat{\section}[block]
  {\normalfont\Large\bfseries}{}{0pt}{\Large}
\titlespacing*{\section}{0pt}{3.5ex plus 1ex minus .2ex}{2.3ex plus .2ex}

\titleformat{\subsection}[block]
  {\normalfont\large\bfseries}{}{0pt}{\large}
\titlespacing*{\subsection}{0pt}{3.25ex plus 1ex minus .2ex}{1.5ex plus .2ex}

\begin{document}
\maketitle
\chapter{Ejercicio 1}
Se confeccionó el siguiente circuito en laboratorio:

\begin{figure}[h]
  \centering
  \begin{minipage}{0.65\textwidth}
    \centering
    \begin{circuitikz}
      % Dibujar la fuente de voltaje
      \draw (0,0) to[battery1, l=\Large$\LEpsilon$, invert] (0,4)
      % Dibujar las resistencias en serie
      to[R, l=$R_1$] (3,4)
      to[R, l=$R_2$] (6,4)
      to[R, l=$R_3$] (9,4)
      % Conectar la última resistencia a tierra y cerrar el circuito
      -- (9,0) -- (0,0);
    \end{circuitikz}
  \end{minipage}\hfill
  \begin{minipage}{0.35\textwidth}
    \centering
    (Valores teoricos)
    $$\LEpsilon = 16V$$
    $$R_1 = 50\Omega$$
    $$R_2 = 35\Omega$$
    $$R_3 = 70\Omega$$
  \end{minipage}
\end{figure}

\section{Mediciones}
Con la fuente desconectada, se midieron los valores de las distintas resistencias individualmente,
y luego la resistencia equivalente del circuito. Obteniendo los siguientes valores:

$$R_1 = 48,5\Omega \hspace{5mm} R_2 = 34,1\Omega \hspace{5mm} R_3 = 69\Omega \hspace{5mm}
R_{eq} = 151,6\Omega$$

Observamos que se cumple la ecuación:
$$R_{eq}=R_1+R_2+R_3$$

Luego, se conectó la fuente al circuito, y se midio la tension entre sus bornes, obteniendo el
valor de:
$$\LEpsilon = 15,43V$$

\section{Cálculos}

Una vez obtenidos estos datos, planteamos la ecuacion de Kirchoff del circuito,
con el objetivo de calcular la corriente y las caidas de tensión en cada resistor.

\begin{minipage}[t]{0.48\textwidth}
  $$
  \begin{aligned}
    0 &= \LEpsilon - V_{R_1} - V_{R_2} - V_{R_3}\\[6pt]
    0 &= \LEpsilon - I (R_1 + R_2 + R_3)\\[6pt]
    I &= \frac{\LEpsilon}{(R_1 + R_2 + R_3)}\\[6pt]
    I &= 101,78A\\[6pt]
  \end{aligned}
  $$
\end{minipage}
\hfill
\begin{minipage}[t]{0.48\textwidth}
  \vspace{7mm}
  $$
  \begin{aligned}
    V_{R_1} &= R_1 \cdot I = 4,9 V\\[6pt]
    V_{R_2} &= R_2 \cdot I = 3,47 V\\[6pt]
    V_{R_3} &= R_3 \cdot I = 7 V\\[6pt]
  \end{aligned}
  $$
\end{minipage}

Observamos que, como era esperado: $\LEpsilon =  V_{R_1} + V_{R_2} + V_{R_3}$. Esto 
indica que el valor de Corriente $(I)$ calculado es correcto.

%$$I_{R_1}=\frac{V_{R_1}}{R_1}$$
%$$I_{R_2}=\frac{V_{R_2}}{R_2}$$
%$$I_{R_3}=\frac{V_{R_3}}{R_3}$$
%
%$$I_{R_1}=I_{R_2}=I_{R_3}=I$$
%
%\section{Teorico}
%$$\epsilon_1 = 16 V$$
%$$R_1=50\Omega$$
%$$R_2=35\Omega$$
%$$R_3=70\Omega$$
%
%$$R_{eq}=155\Omega$$
%$$I = \frac{\epsilon_1}{R_{eq}}$$
%$$I = \frac{16V}{155\Omega}$$
%$$I = 103,23 mA$$
%
%$$V_{R_1} = R_1 I =50\Omega \cdot 103,23 mA= 5,16 V$$
%$$V_{R_2} = R_2 I =35\Omega \cdot 103,23 mA= 3,61 V$$
%$$V_{R_3} = R_3 I =70\Omega \cdot 103,23 mA= 7,23 V$$

\chapter{Ejercicio 2}
Para la segunda experiencia, se armó el circuito a continuación.

\begin{figure}[h]
  \centering
  \begin{minipage}{0.65\textwidth}
      \centering
      \begin{circuitikz}
          % Dibujar la fuente de voltaje
        \draw (0,0) to[battery1, l=$\LEpsilon$, invert] (0,4) -- (2,4)
        to[R, l=$R_1$] (2,0) -- (0,0);
        \draw (2,4) -- (4,4)
          to[R, l=$R_2$] (4,0) -- (0,0);
        \draw (4,4) -- (6,4)
          to[R, l=$R_2$] (6,0) -- (0,0);
      \end{circuitikz}
  \end{minipage}\hfill
  \begin{minipage}{0.35\textwidth}
      \centering
      (Valores teóricos)
      $$\LEpsilon = 16V$$
      $$R_1 = 50\Omega$$
      $$R_2 = 35\Omega$$
      $$R_3 = 70\Omega$$
  \end{minipage}
\end{figure}

\section{Procedimiento}

\subsection{Resistencia equivalente}

Calculamos y medimos las resistencia equivalente del circuito con la fuente desconectada.
Sabemos, de las mediciones del circuito anterior, que los valores reales de las resistencias son:

$$R_1 = 48,5\Omega \hspace{5mm} R_2 = 34,1\Omega \hspace{5mm} R_3 = 69\Omega \hspace{5mm}$$

Entonces:

$$R_{eq}  = \left( \frac{1}{R_1}+\frac{1}{R_2}+\frac{1}{R_3} \right) ^{-1} = 15,51 \Omega$$\\

Mientras que la resistencia equivalente medida en el circuito fue de $R_{eq\hspace{1mm}real}
= 17 \Omega$

\subsection{Corrientes}
Conectamos la fuente de alimentación y medimos la tensión que esta entregaba. Obtuvimos un valor de 
$ \LEpsilon=13,81 V $

Con este valor y el de las resistencias, podemos plantear las ecuaciones de Kirchoff de mallas y
nodos:

\[
\begin{cases}
  N_1 : I_{\LEpsilon} = I_{R_1} + I_{R_1} + I_{R_1} \\
  M_1 : 0 = \LEpsilon - I_{R_1} \cdot R_1 \\
  M_2 : 0 = I_{R_1} \cdot R_1 - I_{R_2} \cdot R_2 \\
  M_3 : 0 = I_{R_2} \cdot R_2 - I_{R_3} \cdot R_3
\end{cases}
\]

Resolviendo este sistema obtendremos el valor de las corritente a travez de cada resistencia

\noindent
\begin{minipage}[t]{0.33\textwidth}
  \centering
  De $M_1$ despejamos:
  $$I_{R1} = \frac{\LEpsilon}{R_1}$$
  $$I_{R1} = 284,74 mA $$
\end{minipage}
\begin{minipage}[t]{0.33\textwidth}
  \centering
  De $M_2$ despejamos:
  $$I_{R2} = \frac{I_{R_1}R_1}{R_2} $$
  $$I_{R2} = 404,98 mA $$
\end{minipage}
\begin{minipage}[t]{0.33\textwidth}
  \centering
  De $M_3$ despejamos:
  $$I_{R2} = \frac{I_{R_2}R_2}{R_3} $$
  $$I_{R3} = 200,14 mA $$
\end{minipage}

\vspace{7mm}

finalmente, a travez de la ecuacion N1, podemos obtener el valor de la corriente entregada 
por la fuente $\LEpsilon$
$$I_{\LEpsilon} = 889,86 mA$$

Obtenidas las corrientes a travez de cada una de las resistencias, podemos calcular su respectiva
caida de tensión:
$$ V_{R_1} = I_{R_1} \cdot R_1 = 13,8V \hspace{10mm} V_{R_2} = I_{R_2} \cdot R_2 = 13,8V 
\hspace{10mm} V_{R_3} = I_{R_3} \cdot R_3 = 13,8V $$

Observamos que la caída de tension es igual en todas las resitencias, lo cual era lo esperado, ya
que estan conectadas en paralelo. De esta forma, pudimos verificar nuestros resultados.

Por último, medimos la corriente entregada por la fuente $\LEpsilon$, observamos que el
valor mensurado es similar al valor calculado de forma teorica.

$$ I_{\LEpsilon \hspace{0,5mm} real} = 820mA \approx 889,86 mA $$

\chapter{Ejercicio 3}

\begin{circuitikz}
    \draw (0,0) to[voltmeter, l=$V_1$, ](0,3) -- (4,3)
    to[R, l=$R2$](6,3) -- (9,3)
    to(9,3) -- (9,4);
    \draw (3,3) -- (3,5)
    to(3,5) -- (4,5)
    to[R, l=$R3$] (6,5) -- (7,5)
    to[ammeter, l =$A_1$] (8,5) -- (9,5)
    to (9,5) -- (9,4);
    \draw (9,4) to[R, l=$R4$](11,4) -- (12,4)
    to [battery1 ,l =$E2$, invert] (14,4) -- (14,1)
    to [R ,l =$R5$](14,-1) -- (14,-3)
    to [ammeter, l=$A_2$](13,-3) -- (12,-3)
    to [R , l =$R6$](10,-3) -- (9, -3)
    to [battery1, l=$E3$](7, -3) -- (6, -3)
    to [R , l =$R7$](4,-3) -- (0, -3)
    to (0,0);
    \draw (3,-3) -- (3,-2)
    to [R ,l =$R_1$](3,-1) -- (3,1)
    to [battery1, l =$E1$, invert] (3,2) -- (3,3);

    \node [above] at (8,0) {$\circlearrowright m_1$};
    \node [above] at (6,3.75) {$\circlearrowright m_2$};
\end{circuitikz}

$$
\begin{aligned}
    &R1=3\Omega \hspace{2cm} &&R2=6\Omega \hspace{2cm} && R3=3\Omega\\[6pt]
    &R4=5\Omega  &&R5=2\Omega  &&R6=1\Omega\\[6pt]
    &R7=2\Omega\\[6pt]
    &E1=50V &&E2=28V  &&A_1=2A \\[6pt]
    &E2=\,? &&V_1=\,? &&A_2=\,?\\[12pt]
\end{aligned}
$$


$$
\left\{
\begin{array}{l}
\text{$I_1= I_2+I_3$} \\[6pt]
\text{$M_1: 0=50v-6\Omega \cdot I_3 - 5\Omega \cdot I_1+\LEpsilon -2\Omega \cdot I_1-1\Omega \cdot I_1-28-2\Omega \cdot I_1 -3\Omega \cdot I_1$} \\[6pt]
\text{$M_2: 0=-3\Omega \cdot I_2 +6\Omega \cdot I_3$}\\[6pt]
\end{array}
\right.
$$
\newpage

\begin{figure}[h]
 \begin{minipage}{0.4\textwidth}
  \centering
    
    $$
    \begin{aligned}
      M_1:\,&0=22V + \LEpsilon -6\Omega \cdot 1A - 13\Omega \cdot 3A\\[6pt]
      &0=22V -45V + \LEpsilon\\[6pt]
      &\LEpsilon = 23V\\[6pt]
      &\phantom{culo}
    \end{aligned}
    $$
  \end{minipage}\hfill
  \begin{minipage}{0.4\textwidth}
    \centering
    $$
    \begin{aligned}
      M_2:\,&0=3\Omega \cdot 2A + 6\Omega \cdot I_3\\[6pt]
      &I_3=\frac{6V}{6\Omega}=1A \\[6pt]
      &I_1=1A+2A=3A \, \rightarrow &&\text{lectura}\\[6pt]
      & &&\text{amperimetro $A_2$}
    \end{aligned}
    $$
  \end{minipage}\hfill
\end{figure}


$$
\begin{aligned}
  &\LEpsilon_2=50v-3\Omega \cdot I_1\\[6pt]
  &\LEpsilon_2=50v -3\Omega \cdot 3A\\[6pt]
  &\LEpsilon_2=50-9v=41V \rightarrow \text{Lectura del multimetro $V_1$}\\
\end{aligned}
$$

\chapter{Ejercicio 4}

\begin{circuitikz}
    \draw (0,0) to[battery1, l =$\LEpsilon_1$, invert](0,3) -- (1,3)
    to[R , l=$R1$](2,3)
    to[ammeter , l=$A_1$](6,3) -- (8,3);
    \draw (8,3) -- (8,4)
    to [R , l =$R2$] (12,4) -- (13,4)
    to (13,4) -- (13,2)
    to [ammeter , l =$A_1$] (11,2)
    to [R , l =$R3$] (9,2)
    to (8,2) -- (8,3);
    \draw (13,3) -- (14,3)
    to [battery1, l =$\LEpsilon_2$] (14,-1.5) -- (14,-2)
    to (14,-2) -- (13,-2);
    \draw (13,-2) -- (13, -3)
    to[R , l=$R5$] (9.5,-3) -- (9,-3)
    to (9,-3) -- (9,-1) 
    to[R, l=$R4$](12.5,-1) -- (13, -1)
    to (13,-2);
    \draw (9,-2) -- (6,-2)
    to [battery1 , l=$\LEpsilon_3$](4,-2)
    to [R, l=$R6$](2,-2)
    to (0,-2) -- (0,0);
    \draw (6,-2)to[voltmeter, l=$V_1$](6,2.75) -- (6,3);
    \draw (1,-2) -- (2,-4)
    to[voltmeter, l=$V_2$](5,-4) -- (6,-2);
\end{circuitikz}

$$
\begin{aligned}
    &R1=5\Omega \hspace{2cm} &&R2=3\Omega \hspace{2cm} && R3=6\Omega\\[6pt]
    &R4=4\Omega  &&R5=12\Omega  &&R6=2\Omega\\[6pt]
    &R7=2\Omega\\[6pt]
    &\LEpsilon_1=50V &&\LEpsilon_2=30V&&\LEpsilon_3=20V\\[6pt]
    &V_1=\,? &&V_2=\,? &&A_1=\,?\\[6pt]
    &A_2=\,?\\[6pt]
\end{aligned}
$$

\end{document}


