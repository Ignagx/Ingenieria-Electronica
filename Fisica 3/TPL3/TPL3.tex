\documentclass[a4paper,12pt]{report}
\usepackage[left=2.5cm,right=2.5cm,top=3cm,bottom=2.5cm]{geometry}
\usepackage{fancyhdr}
\usepackage{etoolbox}
\usepackage{titlesec}
\usepackage{titling} 
\usepackage{pgfplots}
\usepackage{array}
\usepackage{amsmath}
\usepackage{caption}

\pagestyle{fancy}
\fancyhf{} 
\fancyhead[L]{UTN-FRC}
\fancyhead[C]{Fisica Electronica: Caga especifica del electrón}
\fancyhead[R]{2R3}
\renewcommand{\headrulewidth}{0.4pt}
\fancyfoot[C]{\vfill\thepage}
\setlength{\headwidth}{\textwidth} % Hace que el ancho del encabezado coincida con el ancho del texto
\setlength{\headheight}{15pt}  % Ajusta la altura del encabezado
\setlength{\headsep}{20pt}     % Ajusta la separación entre el encabezado y el contenido

\usepackage{titlesec}
\titleformat{\chapter}[display]
  {\normalfont\Large\bfseries}{}{0pt}{}
\titlespacing*{\chapter}{10pt}{-45pt}{10pt}

\usepackage{etoolbox} 
\makeatletter
\patchcmd{\chapter}{\thispagestyle{plain}}{\thispagestyle{fancy}}{}{} %Muestra encabezado en las paginas con \chapter
\makeatother

\titleformat{\chapter}[display]
  {\normalfont\bfseries}{}{0pt}{\huge}
\titlespacing*{\chapter}{0pt}{-30pt}{20pt}

\DeclareMathSizes{12}{13}{6}{5}

\title{
  \fontsize{25}{0}\selectfont Universidad Tecnológica Nacional\\
  \fontsize{22}{30}\selectfont Física Electrónica\\
  \fontsize{18}{25}\selectfont TPL 3: CARGA ESPECIFICA DEL ELECTRON\\
}
\author{
Franco Palombo - 401910\\
Gaston Grasso - 401892\\
Ignacio Gil - 401891\\
Luciano Cortesini - 402719\\
}
\date{24 / 10 / 2024}

\begin{document}

\maketitle

\chapter{Introduccion}

    Una partícula con carga que viaja por el espacio, emite una onda electromagnética a su paso. Un electrón, es una
    partícula con carga negativa, que al ser acelerada por un potencial de aceleración $V$, describe una rapidez $v$.
    Este potencial de aceleración, tambien se puede entender como la energía que tiene el electrón. Esta energía, se
    traduce a energía cinética, haciendo que el electrón describa una velocidad en dirección opuesta a cualquier otra
    carga de su mismo signo:
    \begin{figure}[h!]
        \centering
        \begin{minipage}{0.3\textwidth}
            \begin{equation*}
                \frac{1}{2} m v^2 = eV
            \end{equation*}
        \end{minipage}
        \begin{minipage}{0.3\textwidth}
            \begin{equation}
                \label{v.electron}
                v = \sqrt{\frac{2 eV}{m}}
            \end{equation}
        \end{minipage}
    \end{figure}

    Cuando una partícula cargada se mueve en un campo magnético, sobre ella actúa la fuerza magnética dada por la
    siguiente ecuación:
    \begin{equation}
        \vec{F} = q \vec{v} \times \vec{B}
    \end{equation}
    y su movimiento está determinado por las leyes de Newton.

    En el caso de que el campo magnético sea perpendicular al plano de movimiento de la partícula cargada, esta
    va a describir un movimiento circular uniforme, por tiempo indeterminado, donde $\vec{F}$ va a proporcionar la
    aceleración centripeta para mantener a la partícula girando de manera indeterminada en circulo hasta que se
    remueva el campo magnético o cambie su perpendicularidad.

    Por lo tanto, trayendo de la teoria del movimiento circular uniforme, tenemos que:
    \begin{equation*}
        F = |q| v B = m \frac{v^2}{r}
    \end{equation*}
    donde m es la masa de la partícula. 
    Si volvemos a que nuestra partícula cargada es un electrón, la carga del electrón es $e$, por lo que $|q| = e$.
    Entonces, reemplazando $v$ por la expresión que obtivimos en (\ref{v.electron}), nos queda:
    \begin{equation*}
        e B = m \frac{\sqrt{\frac{2 eV}{m}}}{r}
    \end{equation*}
    Ahora, si despejamos para quedarnos con $\frac{e}{m}$:
    \begin{equation}
        \label{relac.e/m}
        \frac{e}{m} = 2 \frac{V}{(r B)^2}
    \end{equation}

\chapter{Experiencia}
    Para la realización de la experiencia de la medición de la carga específica del electrón se utilizan un dispositivo
    compuesto por dos elementos básicos:
    \begin{itemize}
        \item El tubo de rayos filiformes, que genera los electrones y los acelera bajo la acción de una diferencia de
            potencial.
        \item Un par de bobinas de Helmholtz encargadas de generar el campo magnético uniforme al cual serán sometidos
            los electrones.
    \end{itemize}

% Tabla y gráfico para Radio cm = 5
\begin{table}[h]
    \centering
    \begin{minipage}{0.35\textwidth}
        \centering \captionsetup{labelformat=empty} % sin "Table 1"
        \begin{tabular}{|c|c|}
            \hline
            \textbf{Ih (Amp)} & \textbf{U (Volts)} \\ \hline
            1.59 & 300 \\ \hline
            1.54 & 280 \\ \hline
            1.49 & 260 \\ \hline
            1.44 & 240 \\ \hline
            1.36 & 220 \\ \hline
            1.31 & 200 \\ \hline
            1.24 & 180 \\ \hline
            1.18 & 160 \\ \hline
            1.09 & 140 \\ \hline
            0.98 & 120 \\ \hline
        \end{tabular}
        \caption{Datos para Radio cm = 5}
    \end{minipage}%
    \hfill
    \begin{minipage}{0.55\textwidth}
        \centering
        \centering \captionsetup{labelformat=empty} % sin "Table 1"
        \begin{tikzpicture}
            \begin{axis}[
                xlabel={$ (rB)^2 \cdot 10^{-9}$},
                ylabel={$ 2U $},
                xmin=0, xmax=5,
                ymin=0, ymax=700,
                width=1\textwidth,
                height=0.3\textheight
            ]
            \addplot+[
                only marks,
                mark=*,
                mark size=2pt
            ] coordinates {
                (3.49, 600) (3.27, 560) (3.06, 520) (2.86, 480)
                (2.55, 440) (2.37, 400) (2.12, 360) (1.92, 320)
                (1.64, 280) (1.32, 240)
            };
            \addplot[domain=0:10, samples=2, thick] {168.55 * x + 5.36};
            \end{axis}
        \end{tikzpicture}
        \caption*{Y = 168.55$\cdot$x + 5.36}
    \end{minipage}
\end{table}

% Tabla y gráfico para Radio cm = 4
\begin{table}[h]
    \centering
    \begin{minipage}{0.35\textwidth}
        \centering
        \captionsetup{labelformat=empty} % sin "Table 1"
        \begin{tabular}{|c|c|}
            \hline
            \textbf{Ih (Amp)} & \textbf{U (Volts)} \\ \hline
            1.98 & 300 \\ \hline
            1.90 & 280 \\ \hline
            1.83 & 260 \\ \hline
            1.77 & 240 \\ \hline
            1.69 & 220 \\ \hline
            1.63 & 200 \\ \hline
            1.54 & 180 \\ \hline
            1.44 & 160 \\ \hline
            1.36 & 140 \\ \hline
            1.24 & 120 \\ \hline
        \end{tabular}
        \caption{Datos para Radio cm = 4}
    \end{minipage}%
    \hfill
    \begin{minipage}{0.55\textwidth}
        \centering
        \captionsetup{labelformat=empty} % sin "Table 1"
        \begin{tikzpicture}
            \begin{axis}[
                xlabel={$ (rB)^2 \cdot 10^{-9}$},
                ylabel={$ 2U $},
                xmin=0, xmax=5,
                ymin=0, ymax=700,
                width=1\textwidth,
                height=0.3\textheight
            ]
            \addplot+[
                only marks,
                mark=*,
                mark size=2pt
            ] coordinates {
                (3.46, 600) (3.19, 560) (2.9, 520) (2.76, 480)
                (2.52, 440) (2.34, 400) (2.09, 360) (1.83, 320)
                (1.63, 280) (1.35, 240)
            };
            \addplot[domain=0:10, samples=2, thick] { 175.77 * x -3.08};
            \end{axis}
        \end{tikzpicture}
        \caption*{Y = 175.77$\cdot$x - 3.08}
    \end{minipage}
\end{table}

% Tabla y gráfico para Radio cm = 3
\begin{table}[h]
    \centering
    \begin{minipage}{0.35\textwidth}
        \centering
        \captionsetup{labelformat=empty} % sin "Table 1"
        \begin{tabular}{|c|c|}
            \hline
            \textbf{Ih (Amp)} & \textbf{U (Volts)} \\ \hline
            2.66 & 300 \\ \hline
            2.53 & 280 \\ \hline
            2.44 & 260 \\ \hline
            2.36 & 240 \\ \hline
            2.26 & 220 \\ \hline
            2.16 & 200 \\ \hline
            2.03 & 180 \\ \hline
            1.91 & 160 \\ \hline
            1.80 & 140 \\ \hline
            1.64 & 120 \\ \hline
        \end{tabular}
        \caption{Datos para Radio cm = 3}
    \end{minipage}%
    \hfill
    \begin{minipage}{0.55\textwidth}
        \centering
        \captionsetup{labelformat=empty} % sin "Table 1"
        \begin{tikzpicture}
            \begin{axis}[
                xlabel={$ (rB)^2 \cdot 10^{-9}$},
                ylabel={$ 2U $},
                xmin=0, xmax=5,
                ymin=0, ymax=700,
                width=1\textwidth,
                height=0.27\textheight
            ]
            \addplot+[
                only marks,
                mark=*,
                mark size=2pt
            ] coordinates {
                (3.51, 600) (3.18, 560) (2.96, 520) (2.76, 480)
                (2.53, 440) (2.31, 400) (2.04, 360) (1.81, 320)
                (1.61, 280) (1.33, 240)
            };
            \addplot[domain=0:10, samples=2, thick] {170.09 * x + 11.1};
            \end{axis}
        \end{tikzpicture}
        \caption*{Y = 170.09$\cdot$x + 11.10}
    \end{minipage}
\end{table}

% Tabla y gráfico para Radio cm = 2
\begin{table}[h]
    \centering
        \captionsetup{labelformat=empty} % sin "Table 1"
    \begin{minipage}{0.35\textwidth}
        \centering
        \begin{tabular}{|c|c|}
            \hline
            \textbf{Ih (Amp)} & \textbf{U (Volts)} \\ \hline
            3.91 & 300 \\ \hline
            3.78 & 280 \\ \hline
            3.64 & 260 \\ \hline
            3.54 & 240 \\ \hline
            3.37 & 220 \\ \hline
            3.19 & 200 \\ \hline
            3.04 & 180 \\ \hline
            2.89 & 160 \\ \hline
            2.74 & 140 \\ \hline
            2.43 & 120 \\ \hline
        \end{tabular}
        \caption{Datos para Radio cm = 2}
    \end{minipage}%
    \hfill
    \begin{minipage}{0.55\textwidth}
        \centering
        \captionsetup{labelformat=empty} % sin "Table 1"
        \begin{tikzpicture}
            \begin{axis}[
                xlabel={$ (rB)^2 \cdot 10^{-9}$},
                ylabel={$ 2U $},
                xmin=0, xmax=5,
                ymin=0, ymax=700,
                width=1\textwidth,
                height=0.27\textheight
            ]
            \addplot+[
                only marks,
                mark=*,
                mark size=2pt
            ] coordinates {
                (3.37, 600) (3.15, 560) (2.92, 520) (2.76, 480)
                (2.5, 440) (2.24, 400) (2.04, 360) (1.84, 320)
                (1.65, 280) (1.3, 240)
            };
            \addplot[domain=0:10, samples=2, thick] { 178.09 * x -3.33};
            \end{axis}
        \end{tikzpicture}
        \caption*{Y = 178.09$\cdot$x - 3.33}
    \end{minipage}
\end{table}

\begin{figure}[b]
\centering
\captionsetup{labelformat=empty} % Sin "Table 1"
\begin{tikzpicture}
    \begin{axis}[
        xlabel={$ (rB)^2 \cdot 10^{-9}$},
        ylabel={$ 2U $},
        xmin=0, xmax=5,
        ymin=0, ymax=700,
        width=0.9\textwidth,
        height=0.28\textheight
    ]
    \addplot+[
        only marks,
        mark=*,
        mark size=2pt
    ] coordinates {
        (3.37, 600) (3.15, 560) (2.92, 520) (2.76, 480)
        (2.5, 440) (2.24, 400) (2.04, 360) (1.84, 320)
        (1.65, 280) (1.3, 240)
        (3.49, 600) (3.27, 560) (3.06, 520) (2.86, 480)
        (2.55, 440) (2.37, 400) (2.12, 360) (1.92, 320)
        (1.64, 280) (1.32, 240)
        (3.46, 600) (3.19, 560) (2.9, 520) (2.76, 480)
        (2.52, 440) (2.34, 400) (2.09, 360) (1.83, 320)
        (1.63, 280) (1.35, 240)
        (3.51, 600) (3.18, 560) (2.96, 520) (2.76, 480)
        (2.53, 440) (2.31, 400) (2.04, 360) (1.81, 320)
        (1.61, 280) (1.33, 240)
    };
    \addplot[domain=0:5, samples=2, thick] {172.59 * x + 3.7};
    \end{axis}
\end{tikzpicture}
\caption*{TODOS LOS RADIOS: Y = 175.59$\cdot$x + 3.7 }
\end{figure}

\vspace{5cm}

\chapter{Resultados}

    En las tablas anteriores se presentan los datos obtenidos para diferentes radios. A partir de los gráficos obtenidos, podemos deducir la relación lineal entre $ (rB)^2 $ y $ 2U $. 
    El ajuste de cada uno de los gráficos permite obtener la pendiente que está directamente relacionada con el valor de $ \frac{e}{m} $.

\section{Conclusiones}

    La medición de la carga específica del electrón a través del experimento realizado nos ha permitido observar la relación entre la corriente, el voltaje aplicado y las características del campo magnético. A partir de los datos obtenidos, se puede concluir que el valor de $ \frac{e}{m} $ calculado a partir de los experimentos concuerda con el valor conocido de esta relación, validando así la efectividad de la configuración experimental utilizada.

\end{document}



