\documentclass[a4paper,12pt]{report}
\usepackage[left=2.5cm,right=2.5cm,top=3cm,bottom=2.5cm]{geometry}
\usepackage{fancyhdr}
\usepackage{etoolbox}
\usepackage{titlesec}
\usepackage{titling} 
\usepackage{pgfplots}
\usepackage{array}

\pagestyle{fancy}
\fancyhf{} 
\fancyhead[L]{UTN-FRC}
\fancyhead[C]{FÍSICA ELECTRÓNICA: Caga especifica del electrón}
\fancyhead[R]{2R3}
\renewcommand{\headrulewidth}{0.4pt}
\fancyfoot[C]{\vfill\thepage}
\setlength{\headwidth}{\textwidth} % Hace que el ancho del encabezado coincida con el ancho del texto
\setlength{\headheight}{15pt}  % Ajusta la altura del encabezado
\setlength{\headsep}{20pt}     % Ajusta la separación entre el encabezado y el contenido

\usepackage{titlesec}
\titleformat{\chapter}[display]
  {\normalfont\Large\bfseries}{}{0pt}{}
\titlespacing*{\chapter}{10pt}{-45pt}{10pt}

\usepackage{etoolbox} 
\makeatletter
\patchcmd{\chapter}{\thispagestyle{plain}}{\thispagestyle{fancy}}{}{} %Muestra encabezado en las paginas con \chapter
\makeatother

\titleformat{\chapter}[display]
  {\normalfont\bfseries}{}{0pt}{\huge}
\titlespacing*{\chapter}{0pt}{-30pt}{20pt}

\DeclareMathSizes{12}{13}{6}{5}

\title{
  \fontsize{25}{0}\selectfont Universidad Tecnológica Nacional \\
  \fontsize{22}{30}\selectfont Física Electrónica \\
  \fontsize{18}{25}\selectfont TPL 3: CARGA ESPECIFICA DEL ELECTRON $\frac{E}{M}$\\ 
}
\author{
Franco Palombo\\
Gaston Grasso\\
Ignacio Gil\\
Luciano Cortesini\\
}
\date{25 / 10 / 2024}

\begin{document}

\maketitle

\chapter{Introduccion}

\chapter{Tablas}

% Tabla para Radio cm = 5
\begin{table}[h]
    \centering
    \begin{tabular}{|c|c|}
        \hline
        \textbf{Ih (Amp)} & \textbf{U (Volts)} \\ \hline
        1.59 & 300 \\ \hline
        1.54 & 280 \\ \hline
        1.49 & 260 \\ \hline
        1.44 & 240 \\ \hline
        1.36 & 220 \\ \hline
        1.31 & 200 \\ \hline
        1.24 & 180 \\ \hline
        1.18 & 160 \\ \hline
        1.09 & 140 \\ \hline
        0.98 & 120 \\ \hline
    \end{tabular}
    \caption{Datos para Radio cm = 5}
\end{table}

% Tabla para Radio cm = 4
\begin{table}[h]
    \centering
    \begin{tabular}{|c|c|}
        \hline
        \textbf{Ih (Amp)} & \textbf{U (Volts)} \\ \hline
        1.98 & 300 \\ \hline
        1.90 & 280 \\ \hline
        1.83 & 260 \\ \hline
        1.77 & 240 \\ \hline
        1.69 & 220 \\ \hline
        1.63 & 200 \\ \hline
        1.54 & 180 \\ \hline
        1.44 & 160 \\ \hline
        1.36 & 140 \\ \hline
        1.24 & 120 \\ \hline
    \end{tabular}
    \caption{Datos para Radio cm = 4}
\end{table}

% Tabla para Radio cm = 3
\begin{table}[h]
    \centering
    \begin{tabular}{|c|c|}
        \hline
        \textbf{Ih (Amps)} & \textbf{U (Volts)} \\ \hline
        2.66 & 300 \\ \hline
        2.53 & 280 \\ \hline
        2.44 & 260 \\ \hline
        2.36 & 240 \\ \hline
        2.26 & 220 \\ \hline
        2.16 & 200 \\ \hline
        2.03 & 180 \\ \hline
        1.91 & 160 \\ \hline
        1.80 & 140 \\ \hline
        1.64 & 120 \\ \hline
    \end{tabular}
    \caption{Datos para Radio cm = 3}
\end{table}

% Tabla para Radio cm = 2
\begin{table}[h]
    \centering
    \begin{tabular}{|c|c|}
        \hline
        \textbf{Ih (Amp)} & \textbf{U (Volts)} \\ \hline
      3.91 & 300 \\ \hline
      3.78 & 280 \\ \hline
      3.64 & 260 \\ \hline
      3.54 & 240 \\ \hline
      3.37 & 220 \\ \hline
      3.19 & 200 \\ \hline
      3.04 & 180 \\ \hline
      2.89 & 160 \\ \hline
      2.74 & 140 \\ \hline
      2.43 & 120 \\ \hline
    \end{tabular}
    \caption{Datos para Radio cm = 2}
\end{table}

\end{document}



