\documentclass[a4paper,12pt]{report}
\usepackage[left=2.5cm,right=2.5cm,top=3cm,bottom=2.5cm]{geometry}
\usepackage{fancyhdr}
\usepackage{etoolbox}
\usepackage{titlesec}
\usepackage{titling} 
\usepackage{pgfplots}
\usepackage{array}
\usepackage{amsmath}

\pagestyle{fancy}
\fancyhf{} 
\fancyhead[L]{UTN-FRC}
\fancyhead[C]{Fisica Electronica: Caga especifica del electrón}
\fancyhead[R]{2R3}
\renewcommand{\headrulewidth}{0.4pt}
\fancyfoot[C]{\vfill\thepage}
\setlength{\headwidth}{\textwidth} % Hace que el ancho del encabezado coincida con el ancho del texto
\setlength{\headheight}{15pt}  % Ajusta la altura del encabezado
\setlength{\headsep}{20pt}     % Ajusta la separación entre el encabezado y el contenido

\usepackage{titlesec}
\titleformat{\chapter}[display]
  {\normalfont\Large\bfseries}{}{0pt}{}
\titlespacing*{\chapter}{10pt}{-45pt}{10pt}

\usepackage{etoolbox} 
\makeatletter
\patchcmd{\chapter}{\thispagestyle{plain}}{\thispagestyle{fancy}}{}{} %Muestra encabezado en las paginas con \chapter
\makeatother

\titleformat{\chapter}[display]
  {\normalfont\bfseries}{}{0pt}{\huge}
\titlespacing*{\chapter}{0pt}{-30pt}{20pt}

\DeclareMathSizes{12}{13}{6}{5}

\title{
  \fontsize{25}{0}\selectfont Universidad Tecnológica Nacional\\
  \fontsize{22}{30}\selectfont Física Electrónica\\
  \fontsize{18}{25}\selectfont TPL 3: CARGA ESPECIFICA DEL ELECTRON\\
}
\author{
Franco Palombo\\
Gaston Grasso\\
Ignacio Gil\\
Luciano Cortesini\\
}
\date{25 / 10 / 2024}

\begin{document}

\maketitle

\chapter{Introduccion}

    Una particula con carga que viaja por el espacio, emite una onda electromagnetica a su paso. Un electron, es una
    particula con carga negativa, que al ser acelerada por un potencial de aceleracion $V$, describe una rapidez $v$.
    Este potencial de aceleracion, tambien se puede entender como la energia que tiene el electron. Esta energia, se
    traduce a energia cinetica, haciendo que el electron describa una velocidad en direccion opuesta a cualquier otra
    carga de su mismo signo:
    \begin{figure}[h!]
        \centering
        \begin{minipage}{0.3\textwidth}
            \begin{equation*}
                \frac{1}{2} m v^2 = eV
            \end{equation*}
        \end{minipage}
        \begin{minipage}{0.3\textwidth}
            \begin{equation}
                \label{v.electron}
                v = \sqrt{\frac{2 eV}{m}}
            \end{equation}
        \end{minipage}
    \end{figure}

    Cuando una partícula cargada se mueve en un campo magnético, sobre ella actúa la fuerza magnética dada por la
    siguiente ecuación:
    \begin{equation}
        \vec{F} = q \vec{v} \times \vec{B}
    \end{equation}
    y su movimiento está determinado por las leyes de Newton.

    En el caso de que el campo magnetico sea permendicular al plano de movimiento de la particula cargada, esta
    va a describir un movimiento circular uniforme, por tiempo indeterminado, donde $\vec{F}$ va a proporcionar la
    aceleracion centripeta para mantener a la particula girando de manera indeterminada en circulo hasta que se
    remueva el campo magnetico o cambie su perpendicularidad.

    Por lo tanto, trayendo de la teoria del movimiento circular uniforme, tenemos que:
    \begin{equation*}
        F = |q| v B = m \frac{v^2}{r}
    \end{equation*}
    donde m es la masa de la partícula. 
    Si volvemos a que nuestra particula cargada es un electron, la carga del electron es $e$, por lo que $|q| = e$.
    Entonces, reemplazando $v$ por la expresion que obtivimos en (\ref{v.electron}), nos queda:
    \begin{equation*}
        e B = m \frac{\sqrt{\frac{2 eV}{m}}}{r}
    \end{equation*}
    Ahora, si despejamos para quedarnos con $\frac{e}{m}$:
    \begin{equation}
        \label{relac.e/m}
        \frac{e}{m} = 2 \frac{V}{(r B)^2}
    \end{equation}

\chapter{Experiencia}
    Para la realización de la experiencia de la medición de la carga específica del electrón se utilizan un dispositivo
    compuesto por dos elementos básicos:
    \begin{itemize}
        \item El tubo de rayos filiformes, que genera los electrones y los acelera bajo la acción de una diferencia de
            potencial.
        \item Un par de bobinas de Helmholtz encargadas de generar el campo magnético uniforme al cual serán sometidos
            los electrones.
    \end{itemize}

    % Tabla para Radio cm = 5
    \begin{table}[h]
        \centering
        \begin{tabular}{|c|c|}
            \hline
            \textbf{Ih (Amp)} & \textbf{U (Volts)} \\ \hline
            1.59 & 300 \\ \hline
            1.54 & 280 \\ \hline
            1.49 & 260 \\ \hline
            1.44 & 240 \\ \hline
            1.36 & 220 \\ \hline
            1.31 & 200 \\ \hline
            1.24 & 180 \\ \hline
            1.18 & 160 \\ \hline
            1.09 & 140 \\ \hline
            0.98 & 120 \\ \hline
        \end{tabular}
        \caption{Datos para Radio cm = 5}
    \end{table}
    
    % Tabla para Radio cm = 4
    \begin{table}[h]
        \centering
        \begin{tabular}{|c|c|}
            \hline
            \textbf{Ih (Amp)} & \textbf{U (Volts)} \\ \hline
            1.98 & 300 \\ \hline
            1.90 & 280 \\ \hline
            1.83 & 260 \\ \hline
            1.77 & 240 \\ \hline
            1.69 & 220 \\ \hline
            1.63 & 200 \\ \hline
            1.54 & 180 \\ \hline
            1.44 & 160 \\ \hline
            1.36 & 140 \\ \hline
            1.24 & 120 \\ \hline
        \end{tabular}
        \caption{Datos para Radio cm = 4}
    \end{table}
    
    % Tabla para Radio cm = 3
    \begin{table}[h]
        \centering
        \begin{tabular}{|c|c|}
            \hline
            \textbf{Ih (Amps)} & \textbf{U (Volts)} \\ \hline
            2.66 & 300 \\ \hline
            2.53 & 280 \\ \hline
            2.44 & 260 \\ \hline
            2.36 & 240 \\ \hline
            2.26 & 220 \\ \hline
            2.16 & 200 \\ \hline
            2.03 & 180 \\ \hline
            1.91 & 160 \\ \hline
            1.80 & 140 \\ \hline
            1.64 & 120 \\ \hline
        \end{tabular}
        \caption{Datos para Radio cm = 3}
    \end{table}
    
    % Tabla para Radio cm = 2
    \begin{table}[h]
        \centering
        \begin{tabular}{|c|c|}
            \hline
            \textbf{Ih (Amp)} & \textbf{U (Volts)} \\ \hline
          3.91 & 300 \\ \hline
          3.78 & 280 \\ \hline
          3.64 & 260 \\ \hline
          3.54 & 240 \\ \hline
          3.37 & 220 \\ \hline
          3.19 & 200 \\ \hline
          3.04 & 180 \\ \hline
          2.89 & 160 \\ \hline
          2.74 & 140 \\ \hline
          2.43 & 120 \\ \hline
        \end{tabular}
        \caption{Datos para Radio cm = 2}
    \end{table}

\end{document}



