\documentclass[a4paper,12pt]{report}
\usepackage{adjustbox}
\usepackage[left=2.5cm,right=2.5cm,top=3cm,bottom=2.5cm]{geometry}
\usepackage{fancyhdr}
\usepackage{circuitikz} % Para dibujar circuitos
\usepackage{etoolbox}
\usepackage{multirow}
\usepackage{titlesec}
\usepackage{titling} 
\usepackage{pgfplots}

\pagestyle{fancy}
\fancyhf{} 
\fancyhead[L]{UTN-FRC}
\fancyhead[C]{Fisica Electronica: Efecto Fotoeléctrico}
\fancyhead[R]{2R3}
\renewcommand{\headrulewidth}{0.4pt}
\fancyfoot[C]{\vfill\thepage}
\setlength{\headwidth}{\textwidth} % Hace que el ancho del encabezado coincida con el ancho del texto
\setlength{\headheight}{15pt}  % Ajusta la altura del encabezado
\setlength{\headsep}{20pt}     % Ajusta la separación entre el encabezado y el contenido

\usepackage{titlesec}
\titleformat{\chapter}[display]
  {\normalfont\Large\bfseries}{}{0pt}{}
\titlespacing*{\chapter}{10pt}{-45pt}{10pt}

\usepackage{etoolbox} 
\makeatletter
\patchcmd{\chapter}{\thispagestyle{plain}}{\thispagestyle{fancy}}{}{} %Muestra encabezado en las páginas con \chapter
\makeatother

\titleformat{\chapter}[display]
  {\normalfont\bfseries}{}{0pt}{\huge}
\titlespacing*{\chapter}{0pt}{-30pt}{20pt}

\DeclareMathSizes{12}{13}{6}{5}

\title{%
  \fontsize{25}{0}\selectfont Universidad Tecnológica Nacional \\
  \fontsize{22}{30}\selectfont Física Electrónica \\
  \fontsize{18}{25}\selectfont TPL 2: Efecto Fotoeléctrico
}
\author{
Franco Palombo - 401910\\
Gaston Grasso - 401892\\
Ignacio Gil - 401891\\
Luciano Cortesini - 402719\\
}
\date{16 / 10 / 2024}

\begin{document}

\maketitle

\chapter{Introducción}
  El \textbf{efecto fotoeléctrico} es el fenómeno por el cual ciertos materiales emiten electrones al ser iluminados
  con luz de alta frecuencia. Este proceso ocurre cuando los fotones de la luz inciden sobre el material y transfieren
  su energía a los electrones de este. Para que el electrón sea emitido, la energía del fotón debe ser mayor o igual
  que la \textit{energía de extracción} del material, es decir, la energía mínima necesaria para liberar un electrón de
  su superficie.


  La relación entre la energía del fotón y su frecuencia está dada por la ecuación:

  \[
    E = h \cdot f
  \]

  donde \(E\) es la energía del fotón, \(h\) es la constante de Planck, y \(f\) es la frecuencia de la luz. Si la
  energía del fotón es mayor que la energía de extracción (\(W\)), los electrones son emitidos con una energía cinética
  que se puede expresar como:


  \[
    E_{cin} = h \cdot f - W
  \]

  Es importante destacar que el efecto fotoeléctrico no depende de la intensidad de la luz, sino de la
  \textbf{frecuencia}. Aumentar la intensidad de la luz solo incrementa el número de fotones y, por lo tanto, el número
  de electrones emitidos, pero no su energía cinética.

  El fenómeno es un ejemplo claro de cómo la luz puede comportarse como una partícula, en lugar de como una onda
  continua, y demuestra la naturaleza cuantizada de la interacción entre la radiación electromagnética y la materia.

\chapter{Experiencia de laboratorio}
  Utilizamos una fuente de luz que genera líneas espectrales. Se regulan los vástagos que sostienen la lente convexa
  para que el punto focal este exactamente sobre el cátodo del diodo foto eléctrico en vacío dentro del aparato h/e,
  conectado al circuito \ref{circuito}. Al incidir el haz de luz, se emiten los electrones del cátodo al ánodo y se carga el
  capacitor. Al estabilizarse la tensión medida, significa que ya está cargado el capacitor y la corriente en ese
  momento es nula. El tiempo en el que se carga del condensador y la carga en éste dependerá de la frecuencia que
  incida sobre el cátodo.

  \begin{figure}[h!]
    \hspace{1.2cm}
    \begin{circuitikz}
      \draw (0,0) arc[start angle=90,end angle=-90,radius=0.5 cm];
      \fill (-3mm,-0.5 cm) circle (0.2);
      \draw[very thick] (-27mm,4mm) -- (-17mm,-3mm);
      \draw[very thick] (-17mm,-3mm) -- (-14mm,1mm);
      \draw[very thick] [->] (-14mm,1mm) -- (-5mm,-5mm);
      \draw (0.5cm,-0.5cm) -| (40mm, -20mm) node[pground]{};
      \draw (40mm, 20mm) node[buffer, label={[xshift=-1.2mm, yshift=-2.4mm]{1:1}}]{};
      \draw (-3mm,-0.5cm) |- (35mm, 20mm);
      \draw (47mm,20mm) -- (60mm, 20mm);
      \draw (60mm, 20mm) to[capacitor, -] (60mm,-5.1mm);
      \draw (47mm, 20mm) -- (80mm, 20mm);
      \draw (0.5cm,-0.5cm) -- (80mm,-0.5cm);
      \draw [-|] (80mm, 20mm ) to[voltmeter, -](80mm, -5.1mm);
    \end{circuitikz}
    \caption{Circuito dentro del aparato h/e}
  \end{figure}

  \section{Apartado A}
    Se trabaja con 2 líneas espectrales para investigar la máxima energía de los electrones como una funcion de la
    intensidad. Se descarga por completo el capacitor, y al comenzar el tiempo se incide el haz de luz sobre el cátodo
    del diodo fotoeléctrico en vacío, cargando el capacitor. Al estabilizarse la tensión, se cargó el capacitor
    y el tiempo se frena. Se prueba esto para distintas intensidades de luz, colocando un filtro

    \begin{table}[h!]
      \centering
      \begin{tabular}{|c|c|c|c|}
      \hline
      Linea Espectral & Intensidad [\%] & $V_0$ [mV]& Tiempo [s]\\
      \hline
      \multirow{5}{*}{Verde} & 100 & 368.5 & 53.26 \\
                              \cline{2-4}
                              &  80 & 350.3 & 74.76 \\
                              \cline{2-4}
                              &  60 & 330.6 & 50.54 \\
                              \cline{2-4}
                              &  40 & 303.1 & 43.34 \\
                              \cline{2-4}
                              &  20 & 268.3 & 38.44 \\
      \hline
      \multirow{5}{*}{Amarillo} & 100 & 306.1 & 35 \\
                              \cline{2-4}
                                &  80 & 293.6 & 40.25 \\
                              \cline{2-4}
                                &  60 & 279.7 & 50.20 \\
                              \cline{2-4}
                                &  40 & 261.3 & 40.68 \\
                              \cline{2-4}
                                &  20 & 231 & 31.1 \\
      \hline
      \end{tabular}
      \caption{Cuadro de valores medidos}
    \label{tab:datos experiencia 1}
  \end{table}

  \newpage

  \noindent
  \begin{figure}[h!]
      \noindent % Prevents indentation
      \begin{minipage}{0.4\textwidth} % Left side (first figure)
          \begin{tikzpicture}
              \begin{axis}[
                title={Verde},
                xlabel={Intensidad [$\%$]},
                ylabel={$V_0$ [mV]},
                width=7cm,
                height=7cm,
                yticklabel style={/pgf/number format/fixed,/pgf/number format/precision=3},
                ytick distance = 10,
                xtick distance = 20,
                grid=major,
              ]
              \addplot[green, mark=*] coordinates {
                (100,368.5) (80,350.3) (60,330.6) (40,303.1) (20,268.3)
              };
              \end{axis}
          \end{tikzpicture}
      \end{minipage}
      \hspace{1cm}
      \begin{minipage}{0.4\textwidth} % Right side (second figure)
          \begin{tikzpicture}
              \begin{axis}[
                title={Amarillo},
                xlabel={Intensidad [$\%$]},
                ylabel={$V_0$ [mV]},
                width=7cm,
                height=7cm,
                ytick distance = 10,
                xtick distance = 20,
                grid=major,
              ]
              \addplot[blue, mark=*] coordinates {
                (100,306.1) (80,293.6) (60,279.7) (40,261.3) (20,231)
              };
              \end{axis}
          \end{tikzpicture}
      \end{minipage}
    \end{figure}

    \begin{figure}[h!]
      \begin{minipage}{0.4\textwidth} % Right side (second figure)
          \begin{tikzpicture}
              \begin{axis}[
                title={},
                xlabel={Intensidad [$\%$]},
                ylabel={t [s]},
                width=7cm,
                height=7cm,
                ytick distance = 3,
                xtick distance = 20,
                grid=major,
              ]
              \addplot[green, mark=*] coordinates {
                (100,53.26) (80,74.76) (60,50.54) (40,45.34) (20,38.44)
              };
              \end{axis}
          \end{tikzpicture}
      \end{minipage}
      \hspace{1cm}
      \begin{minipage}{0.4\textwidth} % Right side (second figure)
          \begin{tikzpicture}
              \begin{axis}[
                title={},
                xlabel={Intensidad [$\%$]},
                ylabel={t [s]},
                width=7cm,
                height=7cm,
                ytick distance = 2,
                xtick distance = 20,
                grid=major,
              ]
              \addplot[blue, mark=*] coordinates {
                (100,35) (80,40.25) (60,50.2) (40,40.68) (20,31.1)
              };
              \end{axis}
          \end{tikzpicture}
      \end{minipage}
      \caption{Comparación entre la intensidad y la tensión de corte y tiempo de carga según espectro}
      \label{fig:comparison}
  \end{figure}

  \newpage

  \section{Apartado B}
    Se trabaja con 5 líneas espectrales y 2 órdenes de magnitud. Se obtiene un promedio para la frecuencia ($f$) y 
    para la tension de corte ($V_0$), y asi calcular la constante de Planck ($h$) a partir de la regresion lineal.

    \begin{table}[htbp!]
        \centering
        \begin{adjustbox}{width=\linewidth}
        \begin{tabular}{|c|c|c|c|c|}
        \hline
        \multicolumn{5}{|c|}{Primer Orden} \\
        \hline
        Linea Espectral & Frecuencia [$10^{14} Hz]$ & Longitud de Onda [nm] & $V_0$ [mV] & $eV_0 $\\
        \hline
        Amarillo & 5195 & 577 & 350 & 5.6 $\cdot 10^{-20}$ \\
        \hline
        Verde & 5896 & 546 & 435 & 6.96 $\cdot 10^{-20}$\\
        \hline
        Azul & 6849 & 433.8 & 1100 & 1.76 $\cdot 10^{-19}$\\
        \hline
        Violeta & 7407 & 404.7 & 1302 & 2.083 $\cdot 10^{-19}$\\
        \hline
        Ultravioleta & 8184 & 366.3 & 1030 & 1.648 $\cdot 10^{-19}$\\
        \hline
        \end{tabular}
        \end{adjustbox}
    \end{table}

    \begin{table}[htbp!]
        \centering
        \begin{adjustbox}{width=\linewidth}
        \begin{tabular}{|c|c|c|c|c|}
        \hline
        \multicolumn{5}{|c|}{Segundo Orden} \\
        \hline
        Linea Espectral & Frecuencia [$10^{14} Hz]$ & Longitud de Onda [nm] & $V_0$ [mV]& $eV_0 $\\
        \hline
        Amarillo & 5195 & 577 & 295 & 4.72 $\cdot 10^{-20}$\\
        \hline
        Verde & 5896 & 546 & 322 & 5.152 $\cdot 10^{-20}$\\
        \hline
        Azul & 6849 & 433.8 & 895 & 1.432 $\cdot 10^{-19}$\\
        \hline
        Violeta & 7407 & 404.7 & 965 & 1.544 $\cdot 10^{-19}$\\
        \hline
        Ultravioleta & 8184 & 366.3 & 900 & 1.44 $\cdot 10^{-19}$\\
        \hline
        \end{tabular}
        \end{adjustbox}
    \end{table}

    A partir de los datos obtenidos, se calcula una regresion lineal para cada orden espectral:\\

          \begin{tikzpicture}
\centering
              \begin{axis}[
                title={},
		axis lines = center,
		xlabel={Frecuencia [$[Hz \cdot 10^{18}$]},
		ylabel={$Vo [V]$},
                width=15cm,
                height=7cm,
		ymin = -2, ymax = 3,
                ytick distance = 0.5,
            	xmin = -3e16, xmax = 1e18,
                xtick distance = 1e17,
                grid=major,
              ]
              \addplot[blue, mark=*] coordinates {
		      (5195e14, 0.350) (5896e14, 0.435) (6849e14, 1.100) (7407e14, 1.302) (8184e14, 1.030)
              };
          \addplot[red, very thick][
              domain = -5e16:1.1e19, % Defines the range for x
              samples=20,      % Number of sample points for smoothness
          ]
          {-1.191+3.032e-18 * x};
              \end{axis}
          \end{tikzpicture}

          \begin{tikzpicture}
\centering
              \begin{axis}[
                title={},
		axis lines = center,
		xlabel={Frecuencia [$[Hz \cdot 10^{18}$]},
		ylabel={$Vo [V]$},
                width=15cm,
                height=7cm,
		ymin = -2, ymax = 3,
                ytick distance = 0.5,
            	xmin = -3e16, xmax = 1e18,
                xtick distance = 1e17,
                grid=major,
              ]
              \addplot[blue, mark=*] coordinates {
		      (5195e14, 0.295) (5896e14, 0.322) (6849e14,0.895) (7407e14, 0.965) (8184e14, 0.9)
              };
          \addplot[red, very thick][
              domain = -5e16:1.1e19, % Defines the range for x
              samples=20,      % Number of sample points for smoothness
          ]
          {-1.023+2.533e-18 * x};
              \end{axis}
          \end{tikzpicture}

	  Se obtiene entonces $\frac{h}{e}$ como el promedio de la pendiente de la recta y $\frac{W}{e}$ como el promedio de 
	  la ordenada al origen:
	  $$\frac{h}{e} = 2.783 \cdot 10^{-18} V \cdot s$$
	  $$\frac{W}{e} = -1.107 V $$
	  $$W = -1.771 \cdot 10^{-19} J$$


    Se calcula entonces $h$ y se obtiene el grafico para $V_0$ en funcion de la frecuencia:

    $$h = \frac{W + eV_0}{f}$$
    $$h = 4.453 \cdot 10^{-37} J \cdot s$$


\chapter{Conclusión}

    Los resultados encontrados en el Apartado A reflejan una contradicción a lo expuesto en el documento ya que podemos
  ver que la tensión de corte si varía según la intensidad de la luz. Aun asi, creemos firmemente que esto es producto 
  de:
    \begin{itemize}
      \item Errores en la medición: No era claro cuando estaba cargado el capacitor, ya que nunca se estabilizaba la
      tensión realmente, sino que era más intuición de cuando parar el tiempo.
      \item Instrumentos utilizados mal calibrados: Es muy probable que el lente para bajar la intensidad haya estado
      vieja y descalibrada, provocando no solamente una disminución en la intensidad de la luz, sino que tambien
      puede haber cambiado su dirección y demas variables no deseadas.
    \end{itemize}
  Todo esto influye directamente en las mediciones, obteniendo resultados erróneos. Para evitar tales discrepancias,
  se recomienda siempre utilizar equipamiento del más alto nivel y teniendo precaución a la hora de trabajar con ellos
  para aumentar su vida útil.

  En el Apartado B también se ve resultados bastantes alejados de la realidad. Esto se debe a la imposibilidad de
  aislar de toda radiación al aparato $\frac{h}{e}$.

\end{document}
