\documentclass[a4paper,12pt]{report}
\usepackage[left=2.5cm,right=2.5cm,top=3cm,bottom=2.5cm]{geometry}
\usepackage{fancyhdr}
\usepackage{etoolbox}
\usepackage{titlesec}
\usepackage{titling} 
\usepackage{pgfplots}

\pagestyle{fancy}
\fancyhf{} 
\fancyhead[L]{UTN-FRC}
\fancyhead[C]{FÍSICA ELECTRÓNICA: EFECTO FOTOELECTRICO}
\fancyhead[R]{2R3}
\renewcommand{\headrulewidth}{0.4pt}
\fancyfoot[C]{\vfill\thepage}
\setlength{\headwidth}{\textwidth} % Hace que el ancho del encabezado coincida con el ancho del texto
\setlength{\headheight}{15pt}  % Ajusta la altura del encabezado
\setlength{\headsep}{20pt}     % Ajusta la separación entre el encabezado y el contenido

\usepackage{titlesec}
\titleformat{\chapter}[display]
  {\normalfont\Large\bfseries}{}{0pt}{}
\titlespacing*{\chapter}{10pt}{-45pt}{10pt}

\usepackage{etoolbox} 
\makeatletter
\patchcmd{\chapter}{\thispagestyle{plain}}{\thispagestyle{fancy}}{}{} %Muestra encabezado en las paginas con \chapter
\makeatother

\titleformat{\chapter}[display]
  {\normalfont\bfseries}{}{0pt}{\huge}
\titlespacing*{\chapter}{0pt}{-30pt}{20pt}

\DeclareMathSizes{12}{13}{6}{5}

\title{%
  \fontsize{25}{0}\selectfont Universidad Tecnológica Nacional \\
  \fontsize{22}{30}\selectfont Física Electrónica \\
  \fontsize{18}{25}\selectfont TPL 2: Efecto Fotoelectrico
}
\author{
Franco Palombo\\
Gaston Grasso\\
Ignacio Gil\\
Luciano Cortesini\\
}
\date{17 / 10 / 2024}

\begin{document}

\maketitle

\chapter{Introduccion}
El \textbf{efecto fotoeléctrico} es el fenómeno por el cual ciertos materiales emiten electrones al ser iluminados con luz de alta frecuencia. Este proceso ocurre cuando los fotones de la luz inciden sobre el material y transfieren su energía a los electrones del mismo. Para que el electrón sea emitido, la energía del fotón debe ser mayor o igual que la \textit{energía de extracción} del material, es decir, la energía mínima necesaria para liberar un electrón de su superficie.


La relación entre la energía del fotón y su frecuencia está dada por la ecuación:

\[
E = h \cdot f
\]

donde \(E\) es la energía del fotón, \(h\) es la constante de Planck, y \(f\) es la frecuencia de la luz. Si la energía del fotón es mayor que la energía de extracción (\(W\)), los electrones son emitidos con una energía cinética que se puede expresar como:

\[
E_{cin} = h \cdot f - W
\]

Es importante destacar que el efecto fotoeléctrico no depende de la intensidad de la luz, sino de la \textbf{frecuencia}. Aumentar la intensidad de la luz solo incrementa el número de fotones y, por lo tanto, el número de electrones emitidos, pero no su energía cinética.

El fenómeno es un ejemplo claro de cómo la luz puede comportarse como una partícula, en lugar de como una onda continua, y demuestra la naturaleza cuantizada de la interacción entre la radiación electromagnética y la materia.




\end{document}


