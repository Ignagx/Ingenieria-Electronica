\documentclass[12pt,a4paper]{report}
\usepackage[left=2.5cm,right=2.5cm,top=3cm,bottom=3cm]{geometry}
\usepackage{fancyhdr}
\usepackage{etoolbox}
\usepackage{titlesec}
\usepackage{titling} % para personalizar el título
\usepackage{amssymb}
\usepackage{amsmath}
\usepackage{graphicx}
\usepackage{ulem}
\usepackage{cancel}

\pagestyle{fancy}
\fancyhf{} 
\fancyhead[L]{UTN-FRC}
\fancyhead[C]{ASyS}
\fancyhead[R]{2R3}
\renewcommand{\headrulewidth}{0.4pt}
\fancyfoot[C]{\vfill\thepage}

\patchcmd{\chapter}{\thispagestyle{plain}}{\thispagestyle{fancy}}{}{}

\renewcommand{\chaptername}{Ejercicio}

\titleformat{\chapter}[display]
  {\normalfont\bfseries}{\chaptertitlename\ \thechapter}{15pt}{}
\titlespacing*{\chapter}{0pt}{-30pt}{-30pt}

\DeclareMathSizes{15}{13}{8}{8}

\setlength{\parskip}{6pt}

\title{%
  \fontsize{25}{0}\selectfont Universidad Tecnológica Nacional \\
  \fontsize{22}{30}\selectfont Analisis de Señales y Sistemas \\
  \fontsize{20}{25}\selectfont Trabajo Practico 1
}
\author{
Agustin Suppo Laguia\\
Alejo Agustin Lopez Demichelis\\
Franco Palombo\\
Gaston Grasso\\
Ignacio Gil\\
Jesus Agustin Frigerio\\
Laureano Valentin Reinoso\\
Luciano Tomas Cortesini Perez\\
Matias Gabriel Moran\\
}
\date{13 / 05 / 2024}

\begin{document}
\maketitle

\chapter{}%ejercicio 1
Considerar la ecuación en variable compleja $z^7-jz^7+2^{14}2e^{\frac{j\pi}{2}}=0$.
Obtener el conjunto $S$ de números complejos que solucionan la ecuación.

\textbf{a) Demostrar que  $S = \{z_k = 4e^{j(\frac{\pi}{4}+\frac{2k\pi}{7})} \mid k=-3;-2;-1;0;1;2;3\}$}

\begin{align*}
&z^7 - jz^7 + 2^{14}\sqrt{2}e^{\frac{j\pi}{2}} = 0\\
&z^7(1-j) = 2^{14}\sqrt{2}e^{\frac{-j\pi}{2}}\\
&z^7 = \frac{2^{14}\sqrt{2}e^{-\frac{j\pi}{2}}} {2e^{-\frac{j\pi}{4}}}\\
&z = (2^{14}e^{j(-\frac{\pi} {4})})^{\frac{1}{7}}
\end{align*}
$$z = 4e^{j\frac{(-\frac{\pi}{4}+2k\pi)}{7}} \quad,\quad k = \{-3,-2,-1,0,1,2,3\}$$

$$
S = \left\{ 4e^{j \left( \frac{-25 \pi}{28} \right)}, 4e^{j \left( \frac{-17 \pi}{28} \right)}, 4e^{j \left( \frac{-9 \pi}{28} \right)}, 4e^{j \left( \frac{-\pi}{28} \right)}, 4e^{j \left( \frac{\pi}{4} \right)}, 4e^{j \left( \frac{15 \pi}{28} \right)}, 4e^{j \left( \frac{23 \pi}{28} \right)} \right\}
$$

Vamos a tener 7 soluciones distintas que se repiten en periodos de $2\pi$\\

\textbf{b) Obtener el gráfico de normas del conjunto $S$, esto es:}
    $$k, |k| \quad | \quad k = -3;-2;-1;0;1;2;3 $$
\begin{figure}[h] % Aquí comienza el ambiente de figura
    \centering % Centrar la imagen
    \includegraphics[width=0.6\textwidth]{./Imagenes/foto1Ej1.png} % Insertar la imagen
\end{figure}

\vspace{3cm}
\textbf{c) Obtener el gráfico de argumentos principales del conjunto $S$, esto es:}\\
    $$k, Arg(z_k) \quad | \quad k = -3;-2;-1;0;1;2;3 $$
\begin{figure}[h] % Aquí comienza el ambiente de figura
    \centering % Centrar la imagen
    \includegraphics[width=0.6\textwidth]{./Imagenes/foto2Ej1.png} % Insertar la imagen
\end{figure}
\chapter{}%ejercicio 2
Considerar las siguientes funciones de variable compleja $z = x + jy$:
$$f_1(z) = (4y - y^2 - x - \frac{1}{3}y^3) - j(y + 2x^2 + \frac{1}{3}x^3)$$
$$f_2(z) = (4z - z^2 - z - \frac{1}{3}z^3) - j(z + 2z^2 + \frac{1}{3}z^3)$$

\textbf{a) Obtener la parte real $u_1 = \text{Re}(f_1)$ y la parte imaginaria $v_1 = \text{Im}(f_1)$
, y exponer la igualdad $f_1(z) = u_1 + jv_1$. Similarmente, para la función $f_2(z)$.}\\

$f_1$ parte real e imaginaria
$$u_1=Re(f_1)=4y-y^2-x-\frac{1}{3}y^3$$
$$v_1=Im(f_1)=-y-2x^2-\frac{1}{3}y^3$$

Para obtener la parte real e imaginaria de $f_2$, reemplazamos $z=x+jy$\\
Para el primer término:
$$4z - z^2 - z - \frac{1}{3}z^3=(x+jy) - (x+jy)^2 - (x+jy) - \frac{1}{3}(x+jy)^3$$
$$=4x + 4jy -x^2 -2xyj + y^2 -x -jy - \frac{1}{3}x^3 - jx^2y + xy^2 + \frac{1}{3}jy^3$$
$$=(- \frac{1}{3}x^3 -x^2 + 3x + y^2 + xy^2) +j(3y -2xy + \frac{1}{3}y^3 - x^2y )$$ \\
Para el segundo término:
$$-j (z + 2z^2 + \frac{1}{3}z^3) = -j ((x+jy) + 2(x+jy)^2 + \frac{1}{3}(x+jy)^3)$$
$$=-j (x+jy + 2(x^2+2xyj-y^2) + \frac{1}{3}(x^3 + 3x^2yj + 3x(yj)^2 + (jy)^3))$$
$$=-j ((x + 2x^2 - 2y^2 + \frac{1}{3} x^3 - xy^2) + j( y +4xy + x^2y - \frac{1}{3}y^3))$$
$$=j(-x - 2x^2 + 2y^2 - \frac{1}{3} x^3 + xy^2) + ( y +4xy + x^2y - \frac{1}{3}y^3)$$\\
Luego de separar ambos términos de $f_2$ en su parte real e imaginaria encontraremos la parte real
e imaginaria de $f_2$
$$u_2=Re(f_2)=- \frac{1}{3}x^3 - x^2 + 3x + y^2 + xy^2 + y +4xy + x^2y - \frac{1}{3}y^3$$
$$v_2=Im(f_2)=-x + 3y -2xy - 2x^2 + 2y^2 - x^2y + xy^2 + \frac{1}{3}y^3 - \frac{1}{3} x^3$$

\textbf{b) Obtener el conjunto de números complejos $D_1$ donde la función $f_1(z)$ es derivable. Similarmente, obtener $D_2$ donde $f_2(z)$ es derivable.
¿Hace falta resolver las ecuaciones de Cauchy-Riemann para determinar $D_2$?}

Para obtener el conjunto de números complejos $D_1$ donde la función $f_1(z)$ es derivable, utilizaremos las ecuaciones de Cauchy-Riemman:
$$\frac{\delta u_1}{\delta x}=-1 \quad \quad \frac{\delta v_1}{\delta x}=-4x-x^2$$
$$\frac{\delta v_1}{\delta y}=-1 \quad \quad - \frac{\delta u_1}{\delta y}=-(4-2y-y^2)=y^2+2y-4$$
Si:$$\frac{\delta v_1}{\delta x}=-\frac{\delta u_1}{\delta y}$$
$$-4x-x^2=y^2+2y-4$$
$$x^2+4x+y^2+2y=4$$
$$(x+2)^2+(y+1)^2=9$$\\
Entonces, el conjunto de números complejos $D_1$ donde $f_1$ es derivable es:
$$D_1 = \{z \in \mathbb{C} : |z + 2 + j| = 3\}$$

Para obtener el conjunto de números complejos $D_2$ donde $f_2$ es derivable, no es necesario aplicar las ecuaciones de Cauchy-Riemman,
ya que la función es un polinomio complejo, por lo tanto:
$$D_2=\mathbb{C}$$\\

\textbf{c) Demostrar que $D_1 = \{z \in \mathbb{C} : |z + 2 + j| = 3\}$ y $D_2 = \mathbb{C}$.}

$$|z+2+j|=3$$
$$|(x+jy)+2+j|=3$$
$$|x+2+j(y+1)|=3$$
$$\sqrt{(x+2)^2+(y+1)^2}=3$$
$$(x+2)^2+(y+1)^2=3^2$$\\

\begin{samepage}
\textbf{d) Demostrar que la función $f_1(z)$ no es analítica y que la función $f_2(z)$ es entera.}

\nopagebreak
Al ver los resultados obtenidos por Cauchy-Riemann, observamos que las derivadas parciales de $f_1$ nos dan una condición, en este caso $f_1$ es derivable solamente
en $D1 = \{z \in \mathbb{C} : |z + 2 + j| = 3\}$, o sea una circunferencia de radio $3$ centrada en $(-2, -j)$, sin embargo para cualquier punto de $f$ en la
circunferencia no hay una vecindad o disco abierto alrededor del punto en cuestión en el que se satisfagan las ecuaciones de Cauchy-Riemann.

\nopagebreak
$f_2$ es una función compleja cuyo dominio abarca todo el plano complejo. Dado que $f_2$ es una composición de polinomios, se garantiza la continuidad y
derivabilidad en todo su dominio. Además, al ser derivable en cada punto del plano complejo, $f_2$ es analítica en cada punto del dominio, cumpliendo las
condiciones necesarias para decir que la $f_2$ es una función entera\\
\end{samepage}

\textbf{e) De las funciones anteriores, ¿se puede afirmar que "Derivable implica analítica" o "Analítica implica derivable"? ¿Por qué?}

La derivabilidad no implica analiticidad, ya que, para que una función sea analítica no basta con ser derivable en un punto, sino  que debe serlo en todo un dominio.\\

\textbf{f) De las funciones anteriores, ¿se puede afirmar que $f'_1(z)=\frac{\delta u_1}{\delta x}+j\frac{\delta v_1}{\delta x}$ \quad ó \quad 
$f'_2(z)=\frac{\delta u_2}{\delta x}+j\frac{\delta v_2}{\delta x}$? ¿Por qué?}

Se puede afirmar que, ya que tanto $f_1$ como $f_2$, cumplen las ecuaciones de Cauchy-Riemann y en particular, $f_2$ es una función entera, entonces se puede decir en este caso que 
las derivadas de las funciones son iguales a las derivadas parciales de $u$ y $v$ con respecto a $x$ e $y$

\chapter{}%ejercicio 3

Considerar $$ f(z) = w $$ el mapeo bilineal que transforma los puntos:

\begin{align*}
    z_1 &= \infty, \quad & w_1 &= -j \\
    z_2 &= -j, \quad & w_2 &= 0 \\
    z_3 &= 0, \quad & w_3 &= \infty 
\end{align*}

\textbf{a) Desarrollar la fórmula que determina al mapeo \( f(z) \), empleando razones cruzadas.}

Resolución: Se procede partiendo desde la definición de mapeo para una razón cruzada:

$$ \frac{z - z_1}{z - z_3} \cdot \frac{z_2 - z_3}{z_2 - z_1} = \frac{w - w_1}{w - w_3} \cdot \frac{w_2 - w_3}{w_2 - w_1} $$

Antes de reemplazar los valores de los puntos en la definición anterior, es más cómodo si se simplifica acorde al ejercicio. Debido a que tenemos puntos que tienden al infinito en ambos \( w \) y \( z \), vamos a sacar factor común de aquellos puntos, con el objetivo de simplificarlos, ya sea por cancelación o utilizando las identidades de los números hiperreales:

\begin{align*}
    \frac{z_1 \left(\frac{z}{z_1}-1\right) }{z-z_3} \frac{z_2-z_3}{z_1 \left( \frac{z_3}{z_2} -1 \right) } &= \frac{w-w_1}{w_3 \left( \frac{w}{w_3} - 1 \right)} \frac{w_3 \left( \frac{w_1}{w_3} -1 \right) }{w_2-w_1}\\[10pt]
    \frac{\frac{z}{z_1} -1}{z-z_3} \frac{z_2-z_3}{\frac{z_3}{z_1}-1} &= \frac{w-w_1}{\frac{w}{w_3}-1} \frac{\frac{w_1}{w_3}-1}{w_2-w_1}
\end{align*}
Ahora si, reemplazando los puntos:

$$\frac{\frac{z}{\infty}-1}{z-0} \frac{-j-0}{\frac{0}{\infty}-1} = \frac{w-(-j)}{\frac{w}{\infty}-1} \frac{\frac{-j}{\infty}-1}{0-(-j)}$$

Por las identidades de los números hiperreales, cualquier número dividido una magnitud infinita es igual a cero:

\begin{align*}
    \frac{0-1}{z} \frac{-j}{0-1} &= \frac{w+j}{0-1} \frac{0-1}{j}\\
    \frac{-1}{z} \frac{-j}{-1} &= \frac{w+j}{-1} \frac{-1}{j}\\
    \frac{-1}{z}j &= (-w-j)j\\
    -\frac{1}{z}+j &= -w\\
    w &= \frac{1}{z}-j
\end{align*}

\textbf{b) Obtener las fórmulas y graficar \( R' \) en el plano \( W \), sabiendo que \( R = \{z \in \mathbb{C} \mid \text{Im}(z) - 2\text{Re}(z) = 0\} \) en el plano \( W \).}

Resolución: Para simplificar el desarrollo del ejercicio, se realizarán dos mapeos. Uno, \( w_0 \), que va a ser el mapeo recíproco, y otro, \( w_1 \), que va a ser la traslación del mapeo \( w_0 \) a \( -j \).\\

Para empezar, necesitamos encontrar la relación que tienen las componentes $x$ e $y$ (de $z$) respecto de $u$ y $v$ (de $w_0$):
\begin{align*}
    w_0 &=\frac{1}{z}\\
    u+jv &=\frac{1}{x+jy}\\
    x+jy &= \frac{1}{u+jv}\\
    x+jy &= \frac{1}{u+jv} \frac{u-jv}{u-jv}\\
    x+jy &= \frac{u-jv}{u^2+juv-juv-j^2y^2}\\
    x+jy &= \frac{u}{u^2+v^2}-j\frac{v}{u^2+v^2}\\
    x=\frac{u}{u^2+v^2} \hspace{0.5cm} &;\hspace{0.5cm} y=-\frac{v}{u^2+v^2}\\
\end{align*}


Ya teniendo las relaciones entre \( w_0 \) y \( z \), procedemos a operar la ecuación que define el conjunto, y reemplazar con sus respectivas igualdades las variables que correspondan:

\begin{align*}
&\text{Im}(z) - 2\text{Re}(z) = 0\\ 
&y - 2x = 0\\ 
&-\frac{v}{u^2 + v^2} - 2\frac{u}{u^2 + v^2} = 0\\ 
&-\frac{v}{u^2 + v^2} = 2\frac{u}{u^2 + v^2}\\ 
&-v=2u\\
&-v - 2u = 0\\ 
&-v - 2u = w_0\\ 
\end{align*}

Ya teniendo $w_0$, para completar el mapeo, es tan simple como reemplazar el resultado obtenido en $w_1$ , y se obtiene la ecuación que define el nuevo conjunto $R'$:

$$ w_1 = w_0 - j $$
$$ w_1 = -v - 2u - j $$
$$ R' = \{w \in \mathbb{C} \mid -v - 2u = j\} $$

\begin{figure}[h] % Aquí comienza el ambiente de figura
    \centering % Centrar la imagen
    \includegraphics[width=0.65\textwidth]{./Imagenes/foto1Ej3.png} % Insertar la imagen
\end{figure}

\textbf{c) Obtener las fórmulas y graficar \( C' \) en el plano \( W \), sabiendo que \( C = \{z \in \mathbb{C} \mid |z - 1 - j| = 2\} \) en el plano \( W \).}\\[6pt]

Resolución: Para simplificar el desarrollo del ejercicio, como se hizo antes, se realizarán dos mapeos. Uno, \( w_0 \), que va a ser el mapeo recíproco, y otro, \( w_1 \), que va a ser la traslación del mapeo \( w_0 \) a \( -j \).

Para no repetir, vamos a traer del punto b, las relaciones de \( x \) e \( y \) (de \( z \)) respecto de \( u \) y \( v \) (de \( w_0 \)):

$$ x = \frac{u}{u^2 + v^2}, \quad y = \frac{-v}{u^2 + v^2} $$\\[6pt]
Ahora simplemente obtenemos la ecuación que define el conjunto y reemplazamos correspondientemente:\\
\begin{align*}
|z-1-j| &= 2 \\[6pt]
|x+jy-1-j|  &= 2 \\[6pt]
|x-1+jy-j| &= 2 \\[6pt]
|(x-1)+j(y-1)| &= 2 \\[6pt]
\sqrt{(x-1)^2 + (y-1)^2} &= 2 \\[6pt]
(x-1)^2 + (y-1)^2 &= 2 \\[6pt]
x^2 - 2x + 1 + y^2 - 2y + 1 &= 2 \\[6pt]
x^2 - 2x + y^2 - 2y + 2 &= 2 \\[6pt]
x^2 - 2x + y^2 - 2y &= 0 \\[6pt]
\left(\frac{u}{u^2+v^2} \right)^2 - 2\frac{u}{u^2+v^2} + \left( -\frac{v}{u^2+v^2} \right)^2-2 \left( -\frac{v}{u^2+v^2} \right) &=0\\[6pt]
\frac{u^2}{(u^2+v^2)^2}-2\frac{u}{u^2+v^2} + \frac{v^2}{(u^2+v^2)^2}+2\frac{v}{u^2+v^2}&=0\\[6pt]
\frac{u^2}{(u^2+v^2)^2}+\frac{v^2}{(u^2+v^2)^2}&=2\frac{u}{u^2+v^2} - 2 \frac{v}{u^2+v^2}\\[6pt]
\frac{1}{u^2+v^2} \left( \frac{u^2}{u^2+v^2} + \frac{v^2}{u^2+v^2} \right) &= \frac{1}{u^2+v^2}(2u-2v)\\[6pt]
\frac{u^2+v^2}{u^2+v^2}&=2u-2v\\[6pt]
1 &= 2u - 2v \\[6pt]
w_0 &= 2u - 2v - 1\\[6pt]
\end{align*}
 
Ya teniendo $w_0$, para completar el mapeo, es tan simple como reemplazar el resultado obtenido en $w_1$, y se obtiene la ecuación que define el nuevo conjunto \( C' \):\\
 
 
\begin{align*}
w_1 &= w_0-j \\[5pt]
w_1 &= 2u-2v-1-j \\[5pt]
C' &= \{w \in \mathbb{C} \, | \, 2u-2v-j=1\}
\end{align*}
 
 
\begin{figure}[htbp] % Aquí comienza el ambiente de figura
    \centering % Centrar la imagen
    \includegraphics[width=0.65\textwidth]{./Imagenes/foto2Ej3.png} % Insertar la imagen
\end{figure}

\clearpage

\chapter{}%ejercicio 4

Con este ejercicio se espera que el estudiante controle el comportamiento de un mapeo expresándolo previamente como una composición de mapeos básicos, y como ello, obtener fácilmente las fórmulas de una región en el plano $W$ dadas las fórmulas en el plano $Z$. Considerar el mapeo $w = e^{3z + 2}$.

\textbf{a)} Graficar la región $R = \{z \in \mathbb{C} \, | \, 2 \leq \text{Re}(z) \leq 5; \frac{\pi}{4} \leq \text{Im}(z) \leq \frac{\pi}{2}\}$.

Graficamos la región $R = \{z \in \mathbb{C} \, | \, 2 \leq \text{Re}(z) \leq 5; 4 \leq \text{Im}(z) \leq 2\}$ en el plano $Z$.

Parametrizando la frontera de $R$ en:
\begin{itemize}
    \item $R_1 = x + j4$, $2 \leq x \leq 5$
    \item $R_2 = 5 + jy$, $4 \leq y \leq 2$
    \item $R_3 = x + j2$, $2 \leq x \leq 5$
    \item $R_4 = 2 + jy$, $4 \leq y \leq 2$
\end{itemize}


\begin{figure}[h] % Aquí comienza el ambiente de figura
    \centering % Centrar la imagen
    \includegraphics[width=0.65\textwidth]{./Imagenes/foto1Ej4.png} % Insertar la imagen
\end{figure}

\textbf{b)} Obtener las fórmulas y graficar $R' = \{w = f(z) \, | \, z \in R\}$ en el plano $W$, al ser mapeada desde $R$ en el plano $Z$.

El mapeo dado se puede expresar como una combinación de tres mapeos:

\begin{itemize}
    \item $z \rightarrow 3z$ (ampliación)
    \item $z \rightarrow e^z$ (mapeo exponencial)
    \item $w \rightarrow w + 2$ (traslación)
\end{itemize}

Al aplicar los mapeos, obtenemos en un plano $w$ una región $R$, formada por dos rayos (mapeos de $R_1$ y $R_3$), y por dos arcos de circunferencia (mapeos de $R_2$ y $R_4$).

La nueva región $R'$ está parametrizada por:
\begin{itemize}
    \item $R_1 = e^{3x+j\frac{3\pi}{4}} + 2$, $2 \leq x \leq 5$
    \item $R_2 = e^{3 \cdot 5+j3y} + 2$, $4 \leq y \leq 2$
    \item $R_3 = e^{3x+j\frac{3\pi}{2}} + 2$, $2 \leq x \leq 5$
    \item $R_4 = e^{3 \cdot 5+j3y} + 2$, $4 \leq y \leq 2$
\end{itemize}


\begin{figure}[h] % Aquí comienza el ambiente de figura
    \centering % Centrar la imagen
    \includegraphics[width=0.65\textwidth]{./Imagenes/foto2Ej4.png} % Insertar la imagen
\end{figure}

\chapter{}%ejercicio 5

Considerar la función de variable compleja $f(z) =\frac{z + 3}{(z - 1)^2(z^2 + 9)}$

\textbf{a) Obtener el conjunto de ceros y el conjunto de singularidades de la función.}

Ceros $\rightarrow f(z)=0$

$$f(z)=\frac{z + 3j}{(z - 1)^2(z^2 + 9)}=0$$\\
La función es cero cuando le numerador es cero, por ende:
\begin{align*}
z+3j&=0\\[6pt]
z&=-3j\\[6pt]
\end{align*}
Singularidades:

Como la función es un cociente de polinomios complejos va a tener singularidades cuando el denominador sea cero:

\begin{align*}
    &(z-1)^2(z^2+9)=0\\[6pt]
    &(z-1)^2=0 \hspace{3.5cm}    z^2+9=0\\[6pt]
    &z-1=0     \hspace{4cm}   |z|=\sqrt{-9}\\[6pt]
    &z=1       \hspace{4.75cm}  z=\pm3j\\[6pt]
\end{align*}

El conjunto de singularidades es: $$\{1,3j,-3j\}$$

\textbf{b) Clasificar todas y cada una de las singularidades de la función $f(z)$ en: Evitable, Polo (orden), Esencial.}\\[6pt]
1) Para la singularidad $z_0=1$ calculamos:
$$\lim_{z \to z_0}f(z)=\lim_{z \to 1}\frac{z+3j}{(z - 1)^2(z^2 + 9)}=\lim_{z \to 1}\frac{\cancel{z+3j}}{(z - 1)^2(z-3j)\cancel{(z+3j)}}=$$
$$\lim_{z \to 1}\frac{1}{(z - 1)^2(z-3j)}=\lim_{z \to 1}\frac{1}{0.(1-3j)}=\infty$$
Como el limite es infinito la singularidad $z_0=1$ es un polo. Para calcular el orden, buscamos el exponente $n$ que resuelva
la indeterminación.\\
$$\lim_{z \to z_0}(z-z_0)^n f(z)=\lim_{z \to 1}\cancel{(z - 1)^2}\frac{\cancel{z+3j}}{\cancel{(z - 1)^2}(z-3j)\cancel{(z+3j)}}=$$
$$\lim_{z \to 1}\frac{1}{z-3j}=\frac{1}{1-3j}\rightarrow \text{polo de orden 2}$$
2)De forma similar, para la singularidad $z_1=3j$

$$\lim_{z \to z_1}f(z)=\lim_{z \to 3j}\frac{\cancel{z+3j}}{(z - 1)^2(z-3j)\cancel{(z+3j)}}=\frac{1}{(3j-1)^20}=\infty$$\\
Observamos que esta singularidad también es un polo, de forma análoga al punto anterior, calculamos el orden:
$$\lim_{z \to 3j}\cancel{(z - 3j)}\frac{\cancel{z+3j}}{(z - 1)^2\cancel{(z-3j)}\cancel{(z+3j)}}
=\frac{1}{(3j-1)^2}\rightarrow \text{polo simple}$$\\

3)Finalmente, para clasificar $z_3=-3j$, resolvemos:

$$\lim_{z \to z_3}f(z)=\lim_{z \to -3j}\frac{\cancel{z+3j}}{(z - 1)^2(z-3j)\cancel{(z+3j)}}=\frac{1}{(3j-1)^2(-3j-3j)}$$

Como el limite es finito entonces es una singularidad evitable.\\

\textbf{c) Valuar las siguientes integrales:}\\[6pt]
$$
I_1 = \oint_{|z|=1/2} f(z) \, dz\quad;\quad
I_2 = \oint_{|z|=2} f(z) \, dz\quad;\quad
I_3 = \oint_{|z|=4} f(z) \, dz
$$

\begin{figure}[h]
    \centering
    \begin{minipage}{0.65\textwidth}
        \centering
        \includegraphics[width=\textwidth]{./Imagenes/foto1Ej5.jpeg}
    \end{minipage}\hfill
    \begin{minipage}{0.35\textwidth}
        \centering
        \begin{itemize}
            \item $C_1:|z|=1/2$
            \item $C_2:|z|=2$
            \item $C_3:|z|=4$
            \item $z_1=1$
            \item $z_2=3j$
            \item $z_3=-3j$
        \end{itemize}
    \end{minipage}
\end{figure}

\vspace{1cm}

Primero, aislamos cada singularidad con una curva cerrada $C$ conveniente, y calculamos las integrales para cada una de ellas a través de las formulas de integración 
de Cauchy\\
\begin{samepage}
$$I_{z_1} = \oint_{c}\frac{z+3j}{(z - 1)^2(z-3j)(z+3j)} \, dz =\oint_{c}\frac{\frac{\cancel{z+3j}}{\cancel{(z+3j)}(z-1)^2}}{(z-1)^2} \, dz$$
$$= \oint_{c}\frac{\frac{1}{(z-3j)}}{(z-1)^2} \, dz = \frac{2\pi j}{(2-1)!} \left[ \frac{1}{z-3j} \right]_{z=1}^1= -\frac{2\pi j}{(1-3i)^2}$$\\
$$I_{z_2} = \oint_{c}\frac{z+3j}{(z - 1)^2(z-3j)(z+3j)} \, dz =\oint_{c}\frac{\frac{\cancel{z+3j}}{\cancel{(z+3j)}(z-1)^2}}{z-3j} \, dz $$
$$=\frac{2\pi j}{(3j-1)^2}$$\\
$$I_{z_3} = \oint_{c}\frac{\cancel{z+3j}}{(z - 1)^2(z-3j)\cancel{(z+3j)}} \, dz=0$$
\end{samepage}

Una vez obtenidas, se calculan las integrales $I_1$, $I_2$ y $I_3$ aplicando el teorema de deformación de contorno.

Como no hay singularidades dentro de la curva $|z|=\frac{1}{2}$
$$I_1=\oint_{|z|=\frac{1}{2}}f(z) \, dz= 0$$

Dentro de la curva $|z|=2$, tenemos la singularidad $z_1=1$, por ende:
$$I_2=\oint_{|z|=2}f(z) \, dz = I_{z_1} = - \frac{2 \pi j}{(1-3j)^2}$$

Dentro de la curva $|z|=4$, tenemos las singularidades $z_1=1$, $z_2=3j$ y $z_3=-3j$ entonces:
$$I_3=\oint_{|z|=4}f(z) \, dz = I_{z_1}+I_{z_2}+I_{z_3} =( -\frac{2 \pi j }{(1-3j)^2} + \frac{2 \pi j}{(3j-1)^2} + 0) = 0$$\\

Otro método para resolver estas integrales, es mediante el Teorema del residuo:\\

Para esto, primero se calculamos los residuos de cada singularidad.\\

Para $z_1=1$
$$\text{Res}(f(z),1)=\frac{1}{(2-1)!} \lim_{z \to 1} \frac{d}{\,dz} \left[ \cancel{(z-1)^2} \frac{\cancel{(z+3j)}}{\cancel{(z-1)^2}(z-3j) \cancel{(z+3j)}} \right]$$
$$= \lim_{z \to 1} \frac{d}{\,dz} \left[ \frac{1}{z-3j} \right]=-\frac{1}{(1-3j)^2}$$

Para $z_2=3j$
$$\text{Res}(f(z),3j)= \lim_{z \to 3j} \frac{\cancel{(z-3j)}\cancel{(z+3j)}}{(z-1)^2\cancel{(z-3j)}\cancel{(z+3j)}}= \lim_{z \to 3j} \frac{1}{(z-1)^2}=\frac{1}{(3j-1)^2}$$

Como la singularidad $z_3 = -3j$ es evitable $$\text{Res}(f(z),-3j)=0$$

Finalmente, aplicamos el Teorema del residuo de Cauchy. Al igual que en el método anterior, solo se tienen en cuenta las singularidades que están dentro
de la curva de integración.

$$I_1= \oint_{|z|=\frac{1}{2}}f(z) \, dz = 0 $$

$$I_2= \oint_{|z|=2}f(z) \, dz = 2 \pi j \, \text{Res}(f(z),1)= -\frac{2 \pi j}{(1-3j)^2}$$

$$I_3= \oint_{|z|=4}f(z) \, dz = 2 \pi j \cdot (\text{Res}(f(z),1)+\text{Res}(f(z),3j)+\text{Res}(f(z),-3j))$$
$$=2\pi j \left( -\frac{1}{(1-3j)^2} + \frac{1}{(3j-1)^2} \right)=0$$

Comprobamos así, que ambos caminos son equivalentes.\\

\textbf{d) Para evaluar las integrales del inciso, ¿en qué tenemos que concentrar nuestro análisis?}

Para evaluar las integrales del inciso C tenemos que concentrar nuestro análisis en las singularidades de la función $f(z)$ con respecto a las curvas,
cerradas en este caso.\\

\chapter{}%ejercicio 6

Con este ejercicio se busca que el estudiante obtenga información relevante de la serie de Laurent de una función en variable compleja.
Considerar la función de variable compleja $ f(z) = (z - 1)^2 e^\frac{1}{z - 1} $.\\[6pt]

\textbf{a)  Obtener la serie de Laurent de la función identificando claramente la Parte Analítica y la Parte Singular, esto es,}

$$ f(z) = \sum_{n=0}^{\infty} a_n(z - 1)^n + \sum_{n=1}^{\infty} b_n(z - 1)^{-n} $$

Para obtener la Serie de Laurent de esta función, partimos de la serie de McLaurin de $e^z$
$$e^z=\sum_{k=0}^{\infty}\frac{z^k}{k!}\quad,\quad k \in \mathbb{N}$$
Si realizamos la sustitución:
$$e^\frac{1}{z - 1}=e^u \quad, \quad u=(z - 1)^{-1}$$
Entonces:
$$e^\frac{1}{z - 1}=\sum_{k=0}^{\infty}\frac{(z - 1)^{-k}}{k!}$$
Sustituyendo este resultado en nuestra función
$$f(z) = (z - 1)^2 \sum_{k=0}^{\infty}\frac{(z - 1)^{-k}}{k!}$$
$$f(z) = \sum_{k=0}^{\infty}\frac{(z - 1)^{-k+2}}{k!}$$
$$f(z) = \sum_{k=0}^{\infty}a_k(z - 1)^{-k+2}\qquad, \qquad a_k=\frac{1}{k!}$$

Observamos que el exponente $-k+2$
$$(-k+2) \geq 0 \quad \Leftrightarrow \quad k \leq 2$$
$$(-k+2) < 0 \quad \Leftrightarrow \quad k>2$$
Entonces podemos expresar la función como:
$$f(z) = \sum_{k=0}^{2}a_k(z - 1)^{-k+2} + \sum_{k=3}^{\infty}a_k(z - 1)^{-k+2}\qquad, \qquad a_k=\frac{1}{k!}$$
o lo que es igual:
$$f(z) = \underbrace{\sum_{n=0}^{2}a_n(z - 1)^{n}}_{\text{Parte Analitica}} + \underbrace{\sum_{n=1}^{\infty}b_n(z - 1)^{-n}}_{\text{Parte Singular}}
\qquad, \qquad a_n=\frac{1}{(2-n)!}\quad y \quad b_n=\frac{1}{(n+2)!}$$

\textbf{b)  Teniendo en cuenta la serie del inciso anterior clasificar la singularidad (evitable, polo, 
esencial) y calcular el residuo \( \text{Res}(f, z = 1) \).}

Ya que la parte singular de la serie de Laurent tiene infinitos términos, podemos afirmar que la singularidad
en el punto $1+j0$ es esencial.

Por definición, el residuo es el primer coeficiente de la Parte Singular de la serie de Laurent, por ende:
$$\text{Res}(f, z = 1) = b_1=\frac{1}{(1+2)!} = \frac{1}{6}$$

\textbf{c)  Valuar las siguientes integrales:}

$$ I_1 = \oint_{|z| =\frac{1}{2}} f(z) \, dz; \quad I_2 = \oint_{|z| = 2} f(z) \, dz; \quad I_3 = \oint_{|z| = 4} f(z) \, dz $$

Como que no hay singularidades dentro de la curva $|z|=\frac{1}{2}$
$$\oint_{|z| = 1/2} f(z) \, dz = 0$$

Para la segunda integral, aplicando teorema del residuo
$$\oint_{|z| = 2} f(z) \, dz = 2\pi j \text{Res}(f, z = 1) = 2\pi j\frac{1}{6}=\frac{1}{3}\pi j$$

De igual forma para la tercer integral
$$\oint_{|z| = 4} f(z) \, dz = 2\pi j \text{Res}(f, z = 1) =\frac{1}{3}\pi j$$\\

\textbf{d)  Teniendo en cuenta el inciso anterior podemos afirmar que:}

$$ I_r = \oint_{|z| = r} f(z) \, dz = I_1, \text{ para todo } 0 < r < 1 $$ 
Ya que cualquier curva en este intervalo no va a incluir al único punto singular que tiene nuestra función

Y:
$$ I_r = \oint_{|z| = r} f(z) \, dz = I_2, \text{ para todo } r > 1 $$ 
De manera opuesta, todas estas curvas van a incluir al único punto singular de la función

\chapter{}%ejercicio 7

Con este ejercicio se busca que el estudiante calcule integrales del Análisis I empleando la variable
compleja de forma metodológica.

Calcular el área bajo la curva\\


\begin{figure}[h] % Aquí comienza el ambiente de figura
    \centering % Centrar la imagen
    \includegraphics[width=0.65\textwidth]{./Imagenes/foto1Ej7.png} % Insertar la imagen
\end{figure}

$$a = \int_{0}^{2\pi} \frac{3}{3 - 2\cos(x) + \sin(x)} \, dx$$

siguiendo los siguientes pasos:

\textbf{a)  Defina $z = e^{jx}$ y con ello consiga las siguientes igualdades:}

$$dx = \frac{dz}{jz}; \quad \cos(x) = \frac{z^{2} + 1}{2z}; \quad \sin(x) = \frac{z^{2} - 1}{2jz}$$

Partiendo de :
$$z = e^{jx}$$

y diferenciando:
$$dz = e^{jx}j \, dx$$
$$dz = zj \, dx$$
$$dx = \frac{dz}{zj}$$

Para $\cos(x)$ partimos de:
$$\cos(z)=\frac{e^{jz}+e^{-jz}}{2}$$

Entonces:
$$\cos(x)=\frac{e^{jx}+e^{-jx}}{2}$$

Reemplazado $z=e^{jx}$
$$\cos(x)=\frac{z+z^{-1}}{2}$$
$$\cos(x)=\frac{z}{2}+\frac{z^{-1}}{2}$$
$$\cos(x)=\frac{z}{2}+\frac{1}{2z}$$
$$\cos(x)=\frac{z^2-1}{2z}$$

Similarmente para $\sin(x)$
$$\sin(x)=\frac{e^{jx}-e^{-jx}}{2}$$
$$\sin(x)=\frac{z-z^{-1}}{2j}$$
$$\sin(x)=\frac{z}{2j}-\frac{1}{2jz}$$
$$\sin(x)=\frac{z^2-1}{2jz}$$

\textbf{b)  Realizar una sustitución para obtener una función racional en la variable $z$:}

$$ f(x) \, dx = \frac{3}{3 - 2\frac{z^{2} + 1}{2z} + \frac{z^{2} - 1}{2jz}} \cdot \frac{dz}{jz} = R(z) \, dz $$ 

Y con ello, se valida la igualdad:
$$ \int_{0}^{2\pi} f(x) \, dx = \oint_{|z|=1} R(z) \, dz$$ 

\vspace{6cm}

\textbf{c)  Finalmente, clasificar todas las singularidades de la función racional $R(z)$ y usar el teorema del residuo para calcular el lado derecho de la integral formulada en el inciso anterior.}
Para calcular las singularidades, desarrollamos la función:
$$\frac{3}{3 - 2\frac{z^{2} + 1}{2z} + \frac{z^{2} - 1}{2jz}} \cdot \frac{dz}{jz}$$ 
$$\frac{3}{3 - \frac{2z^{2} + 2}{2z} + \frac{z^{2} - 1}{2jz}} \cdot \frac{dz}{jz}$$ 
$$\frac{3}{3 + \frac{-(2z^{2} + 2)2jz + (z^2-1)2z}{4z^2j}} \cdot \frac{dz}{jz}$$ 
$$\frac{3}{\frac{3(4z^2j) - (2z^{2} + 2)2jz + (z^2-1)2z}{4z^2j}} \cdot \frac{dz}{jz}$$ 
$$\frac{3(4z^2\cancel{j})}{12z^2j - 4jz^{3} - 4jz + 2z^3 - 2z} \cdot \frac{dz}{\cancel{j}z}$$ 
$$\frac{12z^2}{6jz - 2jz^{2} - 2j + z^2 - 1} \cdot \frac{dz}{2z^2}$$ 
$$\frac{6z^2}{(1- 2j)z^{2} + 6jz - (1 + 2j)} \cdot \frac{dz}{z^2}$$ 

Calculamos los ceros de $(1- 2j)z^{2} + 6jz - (1 + 2j)$ por Bashkara

$$\frac{-6j \pm \sqrt{(6j)^2-4(1-2j)(-(1+2j))}} {2(1-2j)} $$
$$\frac{-6j \pm \sqrt{-36-4(-1+2j-2j+4j^2)}} {2-4j} $$
$$\frac{-6j \pm \sqrt{-36-4(-5)}} {2-4j} $$
$$\frac{-6j \pm \sqrt{-16}} {2-4j} $$
$$z_2 = \frac{-6j + 4j} {2-4j} = \frac{2}{5}-\frac{1}{5}j\quad;\quad z_3 = \frac{-6j - 4j} {2-4j} = 2-j$$

Identificamos así los polos:
$$z_1=0\quad;\quad z_2=\frac{2}{5}-\frac{1}{5}j \quad;\quad z_3=2-j$$

\vspace{2cm}

\textbf{Clasificación de polos:}\\[6pt]
Para $z_1$:
$$\lim_{z \to 0}\frac{6\cancel{z^2}}{(1- 2j)z^{2} + 6jz - (1 + 2j)} \cdot \frac{1}{\cancel{z^2}}
=\frac{6}{- (1 + 2j)}=-\frac{6}{5}+\frac{12}{5}j$$ 
\begin{center}\framebox[5cm][c]{\textbf{Singularidad evitable}}\end{center}

Para $z_2$:
$$\lim_{z \to \frac{2}{5}-\frac{1}{5}j}\frac{6z^2}{(1- 2j)z^{2} + 6jz - (1 + 2j)} \cdot \frac{1}{z^2}
$$

\begin{align*}
&\lim_{z \to \frac{2}{5}-\frac{1}{5}j}\,\frac{6\cancel{z^2}}{(6jz-2jz^2-2j+z^2-1)}\frac{1}{\cancel{z^2}}\\
& \lim_{z \to \frac{2}{5}-\frac{1}{5}j}\, \frac{6}{6j\left(\frac{2}{5}-\frac{j}{5} \right)-2j\left(\frac{2}{5}-\frac{j}{5} \right)^2-2j+\left(\frac{2}{5}-\frac{j}{5} \right)^2-1}\\
&\lim_{z \to \frac{2}{5}-\frac{1}{5}j}\,\frac{6}{\frac{6}{5}+\frac{12}{5}j-\frac{8}{25}-\frac{6}{25}j-2j+\frac{3}{25}-\frac{4}{25}j-1}=\frac{6}{0} \quad \framebox[4cm][c]{\textbf{Indeterminación}}\\
\end{align*}

Debido a la indeterminación se procede a determinar el grado del polo utilizando limites\\[8pt]
Para $n=1$:
\begin{align*}
&\lim_{z \to \frac{2}{5}-\frac{1}{5}j}\,\left(z-\left( \frac{2}{5}-\frac{j}{5} \right)\right)\frac{6\cancel{z^2}}{(6jz-2jz^2-2j+z^2-1)}\frac{1}{\cancel{z^2}}=\frac{0}{0} \quad \framebox[3cm][c]{\textbf{L´Hopital}}\\
&\lim_{z \to \frac{2}{5}-\frac{1}{5}j}\,\frac{6}{6j-4jz+2z}\\
&\lim_{z \to \frac{2}{5}-\frac{1}{5}j}\,\frac{6}{6j-4j\left(\frac{2}{5}-\frac{j}{5} \right) +2 \left(\frac{2}{5}-\frac{j}{5} \right)}= -\frac{3}{2}\, j \quad \framebox[4cm][c]{\textbf{Polo simple}}\\
\end{align*}


Para $z_3$:
$$\lim_{z \to 2-j}\frac{6z^2}{(1- 2j)z^{2} + 6jz - (1 + 2j)} \cdot \frac{1}{z^2}
$$

\begin{align*}
&\lim_{z \to 2-j}\,\frac{6\cancel{z^2}}{(6jz-2jz^2-2j+z^2-1)}\frac{1}{\cancel{z^2}}\\
&\lim_{z \to 2-j}\, \frac{6}{6j(2-j)-2j(2-j)^2-2j+(2-j)^2-1}\\
&\lim_{z \to 2-j}\,\frac{6}{6+12j-8-6j-2j+3-4j-1}= \frac{6}{0} \quad \framebox[4cm][c]{\textbf{Indeterminación}}\\
\end{align*}

Debido a la indeterminación se procede a determinar el grado del polo utilizando limites\\[8pt]
Para $n=1$:

\begin{align*}
&\lim_{z \to 2-j}\,(z-(2-j))\frac{6\cancel{z^2}}{(6jz-2jz^2-2j+z^2-1)}\frac{1}{\cancel{z^2}}=\frac{0}{0}\quad \framebox[3cm][c]{\textbf{L´Hopital}}\\
&\lim_{z \to 2-j}\,(z-(2-j))\, {6}{6j-4jz+2z}\\
&\lim_{z \to 2-j}\,(z-(2-j))\, \frac{6}{6j-4j(2-j+2(2-j)}=\frac{3}{2} \, j \quad \framebox[4cm][c]{\textbf{Polo simple}}\\
\end{align*}

Calculamos la integral utilizando el teorema del residuo:

\begin{align*}
& \oint_{|z|=1} \frac{6z^2}{(6jz-2jz^2-2j+z^2-1)}\frac{1}{z^2}\, dz= 2\pi j \sum_{k=1}^1 Res(f(z),zk)\\
&\phantom{\oint_{|z|=1} \frac{6z^2}{(6jz-2jz^2-2j+z^2-1)}\frac{1}{z^2}\, dz}=2\pi j (Res(f(z),\left( \frac{2}{5}-\frac{j}{5} \right)\\
&\phantom{\oint_{|z|=1} \frac{6z^2}{(6jz-2jz^2-2j+z^2-1)}\frac{1}{z^2}\, dz}=2\pi j \left( -\frac{3}{2}\,j \right)\\
&\phantom{\oint_{|z|=1} \frac{6z^2}{(6jz-2jz^2-2j+z^2-1)}\frac{1}{z^2}\, dz}=3\pi = 9,42 \text{U.A}\\
\end{align*}

\end{document}

