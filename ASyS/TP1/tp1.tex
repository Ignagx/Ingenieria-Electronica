\documentclass[12pt]{report}
\usepackage[left=2.5cm,right=2.5cm,top=3cm,bottom=3cm]{geometry}
\usepackage{fancyhdr}
\usepackage{etoolbox}
\usepackage{titlesec}
\usepackage{titling} % para personalizar el título

\pagestyle{fancy}
\fancyhf{} 
\fancyhead[L]{UTN-FRC}
\fancyhead[C]{ASyS}
\fancyhead[R]{2R3}
\renewcommand{\headrulewidth}{0.4pt}
\fancyfoot[C]{\vfill\thepage}

\patchcmd{\chapter}{\thispagestyle{plain}}{\thispagestyle{fancy}}{}{}

\renewcommand{\chaptername}{Ejercicio}

\titleformat{\chapter}[display]
  {\normalfont\bfseries}{\chaptertitlename\ \thechapter}{15pt}{}
\titlespacing*{\chapter}{0pt}{0pt}{0pt}

\DeclareMathSizes{15}{13}{8}{8}

\title{%
  \fontsize{25}{0}\selectfont Universidad Tecnológica Nacional \\
  \fontsize{22}{30}\selectfont Analisis de Señales y Sistemas \\
  \fontsize{20}{25}\selectfont Trabajo Practico 1
}
\author{Luciano Cortesini}
\date{06 / 05 / 2024}

\begin{document}
\maketitle

\chapter{}
Considerar la ecuación en variable compleja $z^7-jz^7+2^{14}2e^{\frac{j\pi}{2}}=0$
Obtener el conjunto $S$ de números complejos que solucionan la ecuación

\begin{enumerate}
  \item Demostrar que  $S = \{z_k = 4e^{j(\frac{\pi}{4}+\frac{2k\pi}{7})} \mid k=-3;-2;-1;0;1;2;3\}$
    $$z^7 - jz^7 + 2^{14}\sqrt{2}e^{\frac{j\pi}{2}} = 0$$
    $$z^7(1-j) = 2^{14}\sqrt{2}e^{\frac{-j\pi}{2}}$$
    $$z^7 = \frac{2^{14}\sqrt{2}e^{-\frac{j\pi}{2}}} {2e^{-\frac{j\pi}{4}}}$$
    $$z = (2^{14}e^{j(-\frac{\pi} {4})})^{1/7}$$
    $$z = 4e^{j\frac{(-\frac{\pi}{4}+2k\pi)}{7}} \quad,\quad k = \{-3,-2,-1,0,1,2,3\}$$

  \item Obtener el gráfico de normas del conjunto $S$, esto es,
    $$k, |k| \quad | \quad k = -3;-2;-1;0;1;2;3 $$

  \item Obtener el gráfico de argumentos principales del conjunto $S$, esto es,
    $$k, Arg(z_k) \quad | \quad k = -3;-2;-1;0;1;2;3 $$

\end{enumerate}

\chapter{}

\end{document}
