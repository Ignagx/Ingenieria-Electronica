\documentclass[12pt]{report}
\usepackage[left=2.5cm,right=2.5cm,top=3cm,bottom=3cm]{geometry}
\usepackage{fancyhdr}
\usepackage{etoolbox}
\usepackage{titlesec}
\usepackage{titling} % para personalizar el título
\usepackage{amssymb}
\usepackage{amsmath}
\usepackage{graphicx}

\pagestyle{fancy}
\fancyhf{} 
\fancyhead[L]{UTN-FRC}
\fancyhead[C]{ASyS}
\fancyhead[R]{2R3}
\renewcommand{\headrulewidth}{0.4pt}
\fancyfoot[C]{\vfill\thepage}

\patchcmd{\chapter}{\thispagestyle{plain}}{\thispagestyle{fancy}}{}{}

\renewcommand{\chaptername}{Ejercicio}

\titleformat{\chapter}[display]
  {\normalfont\bfseries}{\chaptertitlename\ \thechapter}{15pt}{}
\titlespacing*{\chapter}{0pt}{0pt}{0pt}

\DeclareMathSizes{15}{13}{8}{8}

\title{%
  \fontsize{25}{0}\selectfont Universidad Tecnológica Nacional \\
  \fontsize{22}{30}\selectfont Analisis de Señales y Sistemas \\
  \fontsize{20}{25}\selectfont Trabajo Practico 1
}
\author{Luciano Cortesini}
\date{06 / 05 / 2024}

\begin{document}
\maketitle

\chapter{}
Considerar la ecuación en variable compleja $z^7-jz^7+2^{14}2e^{\frac{j\pi}{2}}=0$
Obtener el conjunto $S$ de números complejos que solucionan la ecuación

\begin{enumerate}
  \item Demostrar que  $S = \{z_k = 4e^{j(\frac{\pi}{4}+\frac{2k\pi}{7})} \mid k=-3;-2;-1;0;1;2;3\}$
    $$z^7 - jz^7 + 2^{14}\sqrt{2}e^{\frac{j\pi}{2}} = 0$$
    $$z^7(1-j) = 2^{14}\sqrt{2}e^{\frac{-j\pi}{2}}$$
    $$z^7 = \frac{2^{14}\sqrt{2}e^{-\frac{j\pi}{2}}} {2e^{-\frac{j\pi}{4}}}$$
    $$z = (2^{14}e^{j(-\frac{\pi} {4})})^{\frac{1}{7}}$$
    $$z = 4e^{j\frac{(-\frac{\pi}{4}+2k\pi)}{7}} \quad,\quad k = \{-3,-2,-1,0,1,2,3\}$$

  \item Obtener el gráfico de normas del conjunto $S$, esto es,
    $$k, |k| \quad | \quad k = -3;-2;-1;0;1;2;3 $$

  \item Obtener el gráfico de argumentos principales del conjunto $S$, esto es,
    $$k, Arg(z_k) \quad | \quad k = -3;-2;-1;0;1;2;3 $$

\end{enumerate}

\chapter{}
Considerar las siguientes funciones de variable compleja $z = x + jy$:
\begin{center}
    $f_1(z) = 4y - y^2 - x - 13y^3 - jy + 2x^2 + 13x^3$ \\[10pt]
    $f_2(z) = 4z - z^2 - z - 13z^3 - jz + 2z^2 + 13z^3$
\end{center}

\textbf{a) Obtener la parte real $u_1 = \text{Re}(f_1)$ y la parte imaginaria $v_1 = \text{Im}(f_1)$, y exponer la igualdad $f_1(z) = u_1 + jv_1$. Similarmente, para la función $f_2(z)$.}

\textbf{b) Obtener el conjunto de números complejos $D_1$ donde la función $f_1(z)$ es derivable. Similarmente, obtener $D_2$ donde $f_2(z)$ es derivable. ¿Hace falta resolver las ecuaciones de Cauchy-Riemann para determinar $D_2$?}

\textbf{c) Demostrar que $D_1 = \{z \in \mathbb{C} : |z + 2 + j| = 3\}$ y $D_2 = \mathbb{C}$.}

\textbf{d) Demostrar que la función $f_1(z)$ no es analítica y que la función $f_2(z)$ es entera.}

\textbf{e) De las funciones anteriores, ¿se puede afirmar que "Derivable implica analítica" o "Analítica implica derivable"? ¿Por qué?}

\textbf{f) De las funciones anteriores, ¿se puede afirmar que $f'_1(z) = u_{1x} + jv_{1x}$ o $f'_2(z) = u_{2x} + jv_{2x}$? ¿Por qué?}
\chapter{}

Considerar $$ f(z) = w $$ el mapeo bilineal que transforma los puntos:

\begin{align*}
    z_1 &= \infty, \quad & w_1 &= -j \\
    z_2 &= -j, \quad & w_2 &= 0 \\
    z_3 &= 0, \quad & w_3 &= \infty 
\end{align*}

\textbf{a) Desarrollar la fórmula que determina al mapeo \( f(z) \), empleando razones cruzadas.}

Resolución: Se procede partiendo desde la definición de mapeo para una razón cruzada:

$$ \frac{z - z_1}{z - z_3} \cdot \frac{z_2 - z_3}{z_2 - z_1} = \frac{w - w_1}{w - w_3} \cdot \frac{w_2 - w_3}{w_2 - w_1} $$

Antes de reemplazar los valores de los puntos en la definición anterior, es más cómodo si se simplifica acorde al ejercicio. Debido a que tenemos puntos que tienden al infinito en ambos \( w \) y \( z \), vamos a sacar factor común de aquellos puntos, con el objetivo de simplificarlos, ya sea por cancelación o utilizando las identidades de los números hiperreales:

$$ \frac{z_1z}{z - 1} \cdot \frac{z - z_3}{z_2 - z_3} \cdot \frac{z_2 - z_3}{z_1z_3 - 1} = \frac{w_1w}{w - 1} \cdot \frac{w - w_3}{w_2 - w_3} \cdot \frac{w_2 - w_3}{w_1w_3 - 1} $$

$$ \frac{z}{z - 1} \cdot \frac{-j - 0}{0 - (-j)} \cdot \frac{-j}{-j - 1} = \frac{w}{w - 1} \cdot \frac{\infty - \infty}{0 - \infty} \cdot \frac{0 - \infty}{-j\infty - 1} $$

Por las identidades de los números hiperreales, cualquier número dividido una magnitud infinita es igual a cero:

$$ \frac{0 - 1}{z} \cdot \frac{j}{-j} \cdot \frac{1}{j} = \frac{w + j}{0} \cdot \frac{0 - 1}{-j} \cdot \frac{j}{-1} $$

$$ - \frac{1}{z} \cdot j = - \frac{w + j}{j} $$

$$ w = \frac{1}{z} - j $$

\textbf{b) Obtener las fórmulas y graficar \( R' \) en el plano \( W \), sabiendo que \( R = \{z \in \mathbb{C} \mid \text{Im}(z) - 2\text{Re}(z) = 0\} \) en el plano \( W \).}

Resolución: Para simplificar el desarrollo del ejercicio, se realizarán dos mapeos. Uno, \( w_0 \), que va a ser el mapeo recíproco, y otro, \( w_1 \), que va a ser la traslación del mapeo \( w_0 \) a \( -j \).

Para empezar, necesitamos encontrar la relación que tienen las componentes \( x \) e \( y \) (de \( z \)) respecto de \( u \) y \( v \) (de \( w_0 \)):

$$ w_0 = \frac{1}{z} $$
$$ u + jv = \frac{1}{x + jy} $$
$$ x + jy = \frac{1}{u + jv} $$
$$ x + jy = \frac{1}{u + jv} \cdot \frac{u - jv}{u - jv} $$
$$ x + jy = \frac{u - jv}{u^2 + v^2} $$

$$ x = \frac{u}{u^2 + v^2}, \quad y = \frac{-v}{u^2 + v^2} $$

Ya teniendo las relaciones entre \( w_0 \) y \( z \), procedemos a operar la ecuación que define el conjunto, y reemplazar con sus respectivas igualdades las variables que correspondan:

$$ \text{Im}(z) - 2\text{Re}(z) = 0 $$
$$ y - 2x = 0 $$
$$ -\frac{v}{u^2 + v^2} - 2\frac{u}{u^2 + v^2} = 0 $$
$$ -v - 2u = 0 $$
$$ -v - 2u = w_0 $$

Ya teniendo \( w_0 \), para completar el mapeo, es tan simple como reemplazar el resultado obtenido en \( w_1 \), y se obtiene la ecuación que define el nuevo conjunto \( R' \):

$$ w_1 = w_0 - j $$
$$ w_1 = -v - 2u - j $$

$$ R' = \{w \in \mathbb{C} \mid -v - 2u = j\} $$

\begin{figure}[h] % Aquí comienza el ambiente de figura
    \centering % Centrar la imagen
    \includegraphics[width=0.65\textwidth]{./Imagenes/foto1Ej3.png} % Insertar la imagen
\end{figure}

\textbf{c) Obtener las fórmulas y graficar \( C' \) en el plano \( W \), sabiendo que \( C = \{z \in \mathbb{C} \mid |z - 1 - j| = 2\} \) en el plano \( W \).}\\[6pt]

Resolución: Para simplificar el desarrollo del ejercicio, como se hizo antes, se realizarán dos mapeos. Uno, \( w_0 \), que va a ser el mapeo recíproco, y otro, \( w_1 \), que va a ser la traslación del mapeo \( w_0 \) a \( -j \).

Para no repetir, vamos a traer del punto b, las relaciones de \( x \) e \( y \) (de \( z \)) respecto de \( u \) y \( v \) (de \( w_0 \)):

$$ x = \frac{u}{u^2 + v^2}, \quad y = \frac{-v}{u^2 + v^2} $$\\[6pt]
Ahora simplemente obtenemos la ecuación que define el conjunto y reemplazamos correspondientemente:
\begin{math}
\begin{aligned}
|z-1-j| &= 2 \\[6pt]
|x+jy-1-j| &= 2 \\[6pt]
|x-1+jy-j| &= 2 \\[6pt]
|(x-1)+j(y-1)| &= 2 \\[6pt]
(x-1)^2 + (y-1)^2 &= 2 \\[6pt]
(x-1)^2 + (y-1)^2 &= 2 \\[6pt]
x^2 - 2x + 1 + y^2 - 2y + 1 &= 2 \\[6pt]
x^2 - 2x + y^2 - 2y + 2 &= 2 \\[6pt]
x^2 - 2x + y^2 - 2y &= 0 \\[6pt]
uu^2 + v^2 - 2uu^2 + v^2 - vu^2 + v^2 - 2 - vu^2 + v^2 &= 0 \\[6pt]
u^2 u^2 + v^2 - 2uu^2 + v^2 + v^2 u^2 + v^2 - 2vu^2 + v^2 &= 0 \\[6pt]
u^2 u^2 + v^2 + v^2 u^2 + v^2 &= 2uu^2 + v^2 - 2vu^2 + v^2 \\[6pt]
1 u^2 + v^2 u^2 u^2 + v^2 + v^2 u^2 + v^2 &= 1 u^2 + v^2 2u - 2v \\[6pt]
u^2 + v^2 u^2 + v^2 &= 2u - 2v \\[6pt]
1 &= 2u - 2v \\[6pt]
w_0 &= 2u - 2v - 1\\[6pt]
\end{aligned}
\end{math}

Ya teniendo $w_0$, para completar el mapeo, es tan simple como reemplazar el resultado obtenido en $w_1$, y se obtiene la ecuación que define el nuevo conjunto \( C' \):\\

\begin{math}
\begin{aligned}
    w_1 &= w_0-j \\[5pt]
    w_1 &= 2u-2v-1-j \\[5pt]
    C' &= \{w \in \mathbb{C} \, | \, 2u-2v-j=1\}
\end{aligned}
\end{math}

\begin{figure}[htbp] % Aquí comienza el ambiente de figura
    \centering % Centrar la imagen
    \includegraphics[width=0.65\textwidth]{./Imagenes/foto2Ej3.png} % Insertar la imagen
\end{figure}

\clearpage

\chapter{}

Con este ejercicio se espera que el estudiante controle el comportamiento de un mapeo expresándolo previamente como una composición de mapeos básicos, y como ello, obtener fácilmente las fórmulas de una región en el plano $W$ dadas las fórmulas en el plano $Z$. Considerar el mapeo $w = e^{3z + 2}$.

\textbf{a)} Graficar la región $R = \{z \in \mathbb{C} \, | \, 2 \leq \text{Re}(z) \leq 5; \frac{\pi}{4} \leq \text{Im}(z) \leq \frac{\pi}{2}\}$.

Resolucion:
Graficamos la región $R = \{z \in \mathbb{C} \, | \, 2 \leq \text{Re}(z) \leq 5; 4 \leq \text{Im}(z) \leq 2\}$ en el plano $Z$.

Parametrizando la frontera de $R$ en:
\begin{itemize}
    \item $R_1 = x + j4$, $2 \leq x \leq 5$
    \item $R_2 = 5 + jy$, $4 \leq y \leq 2$
    \item $R_3 = x + j2$, $2 \leq x \leq 5$
    \item $R_4 = 2 + jy$, $4 \leq y \leq 2$
\end{itemize}


\begin{figure}[h] % Aquí comienza el ambiente de figura
    \centering % Centrar la imagen
    \includegraphics[width=0.65\textwidth]{./Imagenes/foto1Ej4.png} % Insertar la imagen
\end{figure}

\textbf{b)} Obtener las fórmulas y graficar $R' = \{w = f(z) \, | \, z \in R\}$ en el plano $W$, al ser mapeada desde $R$ en el plano $Z$.

Resolucion:
El mapeo dado se puede expresar como una combinación de tres mapeos:

\begin{itemize}
    \item $z \rightarrow 3z$ (ampliación)
    \item $z \rightarrow e^z$ (mapeo exponencial)
    \item $w \rightarrow w + 2$ (traslación)
\end{itemize}

Al aplicar los mapeos, obtenemos en un plano $w$ una región $R$, formada por dos rayos (mapeos de $R_1$ y $R_3$), y por dos arcos de circunferencia (mapeos de $R_2$ y $R_4$).

La nueva región $R'$ está parametrizada por:
\begin{itemize}
    \item $R_1 = e^{3x+j\frac{3\pi}{4}} + 2$, $2 \leq x \leq 5$
    \item $R_2 = e^{3 \cdot 5+j3y} + 2$, $4 \leq y \leq 2$
    \item $R_3 = e^{3x+j\frac{3\pi}{2}} + 2$, $2 \leq x \leq 5$
    \item $R_4 = e^{3 \cdot 5+j3y} + 2$, $4 \leq y \leq 2$
\end{itemize}


\begin{figure}[h] % Aquí comienza el ambiente de figura
    \centering % Centrar la imagen
    \includegraphics[width=0.65\textwidth]{./Imagenes/foto2Ej4.png} % Insertar la imagen
\end{figure}

\chapter{}

Considerar la función de variable compleja $f(z) = z + 3j(z - 1)^2(z^2 + 9)$.\\[6pt]

\textbf{a) Obtener el conjunto de ceros y el conjunto de singularidades de la función.}\\[6pt]

\textbf{b) Clasificar todas y cada una de las singularidades de la función $f(z)$ en: Evitable, Polo (orden), Esencial.}\\[6pt]

\textbf{c) Valuar las siguientes integrales:}\\[6pt]
\begin{align*}
    I_1 &= \oint_{|z|=1/2} f(z) \, dz \\[6pt]
    I_2 &= \oint_{|z|=2} f(z) \, dz \\[6pt]
    I_3 &= \oint_{|z|=4} f(z) \, dz\\[6pt]
\end{align*}

\textbf{d) Para evaluar las integrales del inciso, ¿en qué tenemos que concentrar nuestro análisis?}

\chapter{}

Con este ejercicio se busca que el estudiante obtenga información relevante de la serie de Laurent de una función en variable compleja.
Considerar la función de variable compleja \( f(z) = (z - 1)^2 \frac{1}{z - 1} \).\\[6pt]

\textbf{a)  Obtener la serie de Laurent de la función identificando claramente la Parte Analítica y la Parte Singular, esto es,}

$$ f(z) = \sum_{n=0}^{\infty} a_n(z - 1)^n + \sum_{n=1}^{\infty} b_n(z - 1)^{-n} $$

\textbf{b)  Teniendo en cuenta la serie del inciso anterior clasificar la singularidad (evitable, polo, esencial) y calcular el residuo \( \text{Res}(f, z = 1) \).}\\[6pt]

\textbf{c)  Valuar las siguientes integrales:}

$$ I_1 = \oint_{|z| = 1/2} f(z) \, dz; \quad I_2 = \oint_{|z| = 2} f(z) \, dz; \quad I_3 = \oint_{|z| = 4} f(z) \, dz $$

\textbf{d)  Teniendo en cuenta el inciso anterior podemos afirmar que:}

$$ I_r = \oint_{|z| = r} f(z) \, dz = I_1, \text{ para todo } 0 < r < 1 $$ ¿Por qué?

$$ I_r = \oint_{|z| = r} f(z) \, dz = I_2, \text{ para todo } r > 1 $$ ¿Por qué?

\chapter{}

    Con este ejercicio se busca que el estudiante calcule integrales del Análisis I empleando la variable compleja de forma metodológica.\\

Calcular el área bajo la curva\\


\begin{figure}[h] % Aquí comienza el ambiente de figura
    \centering % Centrar la imagen
    \includegraphics[width=0.65\textwidth]{./Imagenes/foto1Ej7.png} % Insertar la imagen
\end{figure}

$a = \int_{0}^{2\pi} \frac{3}{3 - 2\cos(x) + \sin(x)} \, dx$\\[6pt]

siguiendo los siguientes pasos:\\[6pt]

\textbf{a)  Defina $z = e^{x}$ y con ello consiga las siguientes igualdades:}\\[6pt]

$dx = \frac{dz}{jz}; \quad \cos(x) = \frac{z^{2} + 1}{2z}; \quad \sin(x) = \frac{z^{2} - 1}{2jz}$\\[6pt]

\textbf{b)  Realizar una sustitución para obtener una función racional en la variable \( z \):}

$$ f(x) \, dx = \frac{3}{3 - 2\int \left( \frac{z^{2} + 1}{2z} \right) \, dz + \int \left( \frac{z^{2} - 1}{2jz} \right) \, dz} \cdot \frac{dz}{jz} = R(z) \, dz $$ 

Y con ello, se valida la igualdad:

$$ \int_{0}^{2\pi} f(x) \, dx = \oint_{|z|=1} R(z) \, dz$$ 


\textbf{c)  Finalmente, clasificar todas las singularidades de la función racional $R(z)$ y usar el teorema del residuo para calcular el lado derecho de la integral formulada en el inciso anterior.}


\end{document}
